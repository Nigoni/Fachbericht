\section{Einleitung}


%Relevanz des Themas und Motivation


%Problembeschreibung


%Zielsetzung


%Methoden


%Aufbau der Bachelorarbeit

In der Leistungselektronik werden ein- und dreiphasige Thyristorsteller seit Jahren für verschiedenste Anwendungen eingesetzt. Dabei gibt es zwei beliebte Methoden wie die Ansteuerung der Steller erfolgen kann. Die Phasenanschnitt-Steuerung ist eine geläufige Vorgehensweise, um die Leistung von ohmsche Lasten zu regulieren. Da bei dieser Steuerung jedoch harmonische Schwingungen (Vielfaches der Netzfrequenz) auftreten, sind ihr vom Netzbetreiber einige Grenzen gesetzt. Ein Beispiel dafür sind die zulässigen Höchstwerten der Oberschwingungsströme bei den verschiedenen Ordnungen. Eine andere Möglichkeit, die verwendet wird, um Spannung und Strom zu regulieren, ist eine Schwingungspaket-Steuerung. Die Leistung wird dabei während mehreren Netzperioden voll bezogen und dann wieder weggeschaltet. Dieses strikte Zu- und Wegschalten des Verbrauchers vom Netz erzeugt neben den harmonischen, auch subharmonische Schwingungen (Bruchteile der Netzfrequenz). Die beiden Effekte sind im Netz unerwünscht, da die ordnungsmässige Funktion der Betriebsmittel beeinträchtigen und zu einer Verschmutzung des Netzes führen.
Eine alternative Variante, die die oben genannten Probleme vermindern könnten, wäre eine Kombination der beiden Steuerungsarten. Dabei würde man mit Hilfe der Phasenanschnittsteuerung das sanfte Hoch- und Runterfahren der Spannung regulieren. Zusätzlich würde mit einer solchen Schwingungspaketsteuerung die Anzahl Pakete des Auf- und Absteuerns eingeschaltet werden.\\
Das Ziel dieser Bachelorarbeit ist es, ein solches Verfahren zu entwerfen und zu analysieren. Der Vergleich mit den bereits vorhandenen Steuerungsarten soll zeigen, dass eine Kombination der beiden Verfahren genutzt werden kann, um die verlustbehafteten Oberschwingungen zu minimieren. Ein geeigneter Messaufbau aller Varianten ist von Nutzen, damit man einen visuellen Vergleich hat.\\
Als Erstes werden jedoch zuerst alle relevanten Informationen über die gängigen Steuerverfahren für ein- und dreiphasige Wechselspannungssteller eruiert. Anschliessend sollen analytisch die harmonischen Oberschwingungen in Funktion zum Zündwinkel der Phasenanschnitt-Steuerung (ein- und dreiphasig) bestimmt werden. Auch die Harmonische in Funktion des Ein- und Ausschalt-Verhältnisses für Schwingungspaket-Steuerung sind ein wichtiger Bestandteil dieser Arbeit. Die Resultate aller Steuerungsverfahren, die man mit den Simulations-Tools Plecs und Matlab aufgebaut hat, werden mit einem geeigneten Laboraufbau verglichen. Danach betrachtet man ein Verfahren mit sanftem Hoch- und Runterfahren der Leistung und kontrolliert das Ganze messtechnisch mit den dazugehörigen erstellten Simulationen. Mit Hilfe eines Arduinos und einer kleinen Software kann schlussendlich die neu entwickelte Ansteuerung, bei einer ohmschen Last und einem Asynchronmotor getestet werden. Bei den gefundenen Verfahren muss immer darauf geachtet werden, dass sie zwingend die Netzvorschriften einhalten.\\
Die vorliegende Projektarbeit gliedert sich in 5 Kapitel. Im ersten Teil werden die wichtigsten Grundlagen erläutert. Dies sind die Funktionalität der einzelnen behandelten Steuerungsarten, die damit auftretenden Probleme und die Formeln die für die Berechnungen der verschiedenen Spektren, verwendet wurden. Ausserdem werden die wichtigsten Normen kurz zusammengefasst. Im Kapitel Simulation sind alle Resultate der simulierten Verfahren, welche mit Plecs und Matlab dargestellt sind, aufgezeigt. Des weiteren verglich man die gegenseitigen Resultate der Simulationen und konnte sie so verifiziert. Das 4. Kapitel beinhaltet den Messaufbau sowie alle Komponenten, die verwendet wurden, um einen geeigneten Laboraufbau zu erstellen. Anschliessend sind im Kapitel 5 die Resultate des Messaufbaus vorgestellt. Am Ende folgt im Kapitel 6 die Diskussion zu den verschiedenen Ansteuerungsverfahren und es wird ein endgültiges Fazit gezogen.





















