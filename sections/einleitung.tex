\section{Einleitung}


%Relevanz des Themas und Motivation


%Problembeschreibung


%Zielsetzung


%Methoden


%Aufbau der Bachelorarbeit

In der Leistungselektronik werden ein- und dreiphasige Thyristorsteller seit Jahren für verschiedenste Anwendungen eingesetzt. Zum Beispiel für die Leistungsregulierung in Dimmern bei Lampen oder auch die Regelung von Heizelementen und anderen ohmsche Lasten. Dabei gibt es zwei beliebte Methoden, wie die Ansteuerung der Steller erfolgen kann. Die Phasenanschnitt-Steuerung ist eine geläufige Vorgehensweise, um die Leistung von ohmschen Lasten zu regulieren. Da bei dieser Steuerung jedoch harmonische Schwingungen (Vielfaches der Netzfrequenz) auftreten, sind ihnen vom Netzbetreiber einige Limiten gesetzt. Ein Beispiel dafür sind die zulässigen Höchstwerten der Oberschwingungsströme bei den verschiedenen Ordnungen. Eine andere Möglichkeit, die verwendet wird, um Spannung und Strom zu regulieren, ist die Schwingungspaketsteuerung. Die Leistung wird dabei während mehreren Netzperioden voll bezogen und dann wieder weggeschaltet. Dieses strikte Zu- und Wegschalten des Verbrauchers vom Netz erzeugt neben den harmonischen, auch subharmonische Schwingungen (Bruchteile der Netzfrequenz). Diese Effekte sind im Netz unerwünscht, da die ordnungsmässige Funktion der Betriebsmittel beeinträchtigen und zu nicht sinus förmigen Ströme und Spannungen führen kann.
Eine alternative Variante, die die oben genannten Probleme vermindern könnte, wäre eine Kombination der beiden Steuerungsarten. Dabei würde man mithilfe der Phasenanschnittsteuerung das sanfte Hoch- und Runterfahren der Spannung regulieren. Zusätzlich würde mit einer solchen Schwingungspaketsteuerung die Anzahl Pakete des Auf- und Absteuerns eingeschaltet werden.\\
Das Ziel dieser Bachelorarbeit ist es, ein solches Verfahren zu entwerfen und zu analysieren. Der Vergleich mit den bereits vorhandenen Steuerungsarten soll zeigen, dass eine Kombination der beiden Verfahren genutzt werden kann, um die verlustbehafteten Oberschwingungen zu minimieren. Mit einem geeigneten Messaufbau aller Varianten können die Unterschiede demonstriert werden.\\
Zunächst werden jedoch alle relevanten Informationen über die gängigen Steuerverfahren für ein- und dreiphasige Wechselspannungssteller zusammengetragen. Anschliessend sollen analytisch die harmonischen Oberschwingungen in Funktion zum Zündwinkel der Phasenanschnitt-Steuerung (ein- und dreiphasig) bestimmt werden. Auch die Harmonische in Funktion des Ein- und Ausschalt-Verhältnisses für Schwingungspaket-Steuerung sind ein wichtiger Bestandteil dieser Arbeit. Die Resultate aller Steuerungsverfahren, die man mit den Simulationstools Plecs und Matlab erarbeitet hat, werden mit den Ergebnissen aus einem geeigneten Laboraufbau verglichen. Danach betrachtet man ein Verfahren mit sanftem Hoch- und Runterfahren der Leistung und verifiziert die zugehörigen Simulationen mit den Messungen. Mithilfe eines Arduinos und einer kleinen Software kann die neu entwickelte Ansteuerung bei einer ohmschen Last und einem Asynchronmotor getestet werden. Schlussendlich soll herausgefunden werden, ob es möglich ist eine Sparvariante mit dem externen Thyristorsteller zu implementieren. Bei den gefundenen Verfahren muss immer darauf geachtet werden, dass sie zwingend die Netzvorschriften einhalten.\\
Die vorliegende Projektarbeit gliedert sich in 5 Kapitel. Im ersten Teil werden die wichtigsten Grundlagen erläutert. Dies sind die Funktion der einzelnen behandelten Steuerungsarten, die damit auftretenden Probleme und die Formeln, die für die Berechnungen der verschiedenen Spektren, verwendet wurden. Ausserdem werden die wichtigsten Normen kurz zusammengefasst. Im Kapitel Simulation sind alle Resultate der simulierten Verfahren, welche mit Plecs und Matlab dargestellt sind, aufgezeigt. Zusätzlich wurden die Resultate der beiden Simulationsprogrammen miteinander verglichen und gegenseitig verifiziert. Das 4. Kapitel beinhaltet den Messaufbau sowie alle Komponenten, die verwendet wurden, um einen geeigneten Laboraufbau zu erstellen. Anschliessend sind im Kapitel 5 die Resultate des Messaufbaus vorgestellt. Am Ende folgt im Kapitel 6 die Diskussion zu den verschiedenen Ansteuerungsverfahren und es wird ein abschliessendes Fazit gezogen.





















