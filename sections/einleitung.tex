\section{Einleitung}


%Relevanz des Themas und Motivation


%Problembeschreibung


%Zielsetzung


%Methoden


%Aufbau der Bachelorarbeit

In der Leistungselektronik sind ein- und dreiphasige Thyristorsteller seit Jahren beliebte Methoden für die Ansteuerungen von verschiedene Verbrauchern. Die Phasenanschnitt-Steuerung ist eine geläufige Vorgehensweise um die Leistung von ohmsche Lasten zu regulieren. Da bei dieser Steuerung jedoch Harmonische Schwingungen (Vielfaches der Netzfrequenz) auftreten, sind ihr vom Netzbetreiber einige Grenzen gesetzt. Ein Beispiel dafür sind die zulässigen Höchstwerten der Oberschwingungsströme bei den verschiedenen Ordnungen. Eine andere Möglichkeit die verwendet wird, um Spannung und Strom zu regulieren, ist eine Schwingungspaket-Steuerung. Die Leistung wird dabei während mehreren Netzperioden voll bezogen und dann wieder weggeschaltet. Dieses strikte Zu- und Wegschalten des Verbrauchers vom Netz, erzeugen neben den Harmonischen, auch Subharmonische Schwingungen (Bruchteile der Netzfrequenz). Die beiden Effekte sind im Netz unerwünscht, da diese die ordnungsmässige Funktion der Betriebsmittel beeinträchtigen und zu einer Verschmutzung des Netzes führen.
Eine alternative Variante die die oben genannten Probleme vermindern könnten, wäre eine Kombination der beiden Steuerungsarten. Dabei würde man mit Hilfe der Phasenanschnittsteuerung das sanfte Hoch- und Runterfahren der Spannung regulieren. Zusätzlich wird mit der Schwingungspaketsteuerung die Anzahl Pakete mit dem Sanft-Anlasser eingeschaltet.\\
Das Ziel dieser Bachelorarbeit ist es, ein solches Verfahren zu entwerfen und zu analysieren. Der Vergleich mit den bereits vorhandenen Steuerungsarten soll zeigen, dass ein Kombination der beiden Verfahren genutzt werden kann, um die verlustbehafteten Oberschwingungen zu minimieren. Ein geeigneter Messaufbau aller Varianten ist von nutzten, damit man den Vergleich auch visuell darstellen kann. Als erstes werden jedoch zuerst alle relevanten Informationen über die gängigen Steuerverfahren für ein- und dreiphasige Wechselspannungssteller eruiert. Anschliessend sollen analytisch die Harmonischen Oberschwingungen in Funktion zum Zündwinkel der Phasenanschnitt-Steuerung (ein- und dreiphasig) bestimmt werden. Auch die Harmonische in Funktion des Ein- und Ausschalt-Verhältnisses für Schwingungspaket-Steuerung sind ein wichtiger Bestandteil dieser Arbeit. Die Resultate aller Steuerungsverfahren, die man mit den Simulations-Tools Plecs und Matlab aufgebaut hat, werden mit einem geeigneten Laboraufbau verglichen. Das Ziel dieser Methoden sind die Verifizierung der Ergebnisse. Danach betrachtet man ein Verfahren mit sanftem Hoch- und Runterfahren der Leistung und kontrolliert das Ganze messtechnisch mit den dazugehörigen erstellten Simulationen. Mit Hilfe eines Arduinos und einer kleinen Software, kann schlussendlich die neu entwickelte Ansteuerung, bei einer ohmschen Last und einem Asynchronmotor getestet werden. Danach kann ein endgültiges Fazit des Steuerverfahrens gezogen werden. Bei den gefundenen Verfahren muss man ausserdem immer darauf achten, dass sie zwingend die Netzvorschriften einhalten.\\
Die vorliegende Projektarbeit gliedert sich in 5 Hauptkapitel. Im ersten Teil werden die wichtigsten Grundlagen erläutert. Sie dienen als Anfangskenntnisse der Arbeit und beschreiben die Funktionalität der einzelnen behandelten Steuerungsarten, die auftretenden Probleme und die Gegenmassnahmen die man dabei treffen kann. Ausserdem werden die wichtigsten Normen, die betrachtet wurden, kurz zusammengefasst. Im nächsten Kapitel Simulation sind alle Resultate der simulierten Verfahren, welche man mit Plecs und Matlab dargestellt hat, aufgezeigt. Des weiteren verglich man die gegenseitigen Resultate der Simulationen und konnte sie so verifiziert. Das darauffolgende Kapitel 4 Messaufbau zeigen alle Komponenten, die verwendet wurden um einen geeigneten Laboraufbau erstellen zu können. Anschliessend, im Kapitel 5 Messaufbau Resultate sind die Ergebnisse des Laboraufbaus dargestellt. Am Ende folgt im Kapitel 6 die Diskussion zu den verschiedenen Ansteuerungsverfahren und zeigt das endgültige Fazit. Des weiteren wird beschrieben inwiefern die gesetzten Ziele erreicht wurden. Bei der Wiederaufnahme des Projektes soll der Bericht eine verständliche Hilfe darstellen.





















