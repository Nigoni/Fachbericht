\section{Einleitung}


%Relevanz des Themas und Motivation


%Problembeschreibung


%Zielsetzung


%Methoden


%Aufbau der Bachelorarbeit

In der Leistungselektronik sind ein- und dreiphasige Thyristorsteller seit Jahren beliebte Methoden für die Ansteuerungen von verschiedene Verbrauchern. Die Phasenanschnitt-Steuerung ist eine geläufige Vorgehensweise um die Leistung von ohmsche Lasten zu regulieren. Da bei dieser Steuerung jedoch Harmonische (Vielfaches der Netzfrequenz) Schwingungen auftreten, sind ihr vom Netzbetreiber einige Grenzen gesetzt. Eine andere Möglichkeit die verwendet wird um Spannung und Strom zu regulieren, ist eine Schwingungspaket-Steuerung. Die Leistung wird dabei während mehreren Netzperioden voll bezogen und dann wieder weggeschaltet. Dieses strikte Zu- und Wegschalten des Verbrauchers vom Netz, erzeugen neben den Harmonischen auch Subharmonische (Bruchteile der Netzfrequenz) Oberschwingungen.\\
Eine weitere Variante die die oben genannten Probleme vermindern könnte, wäre eine Kombination der beiden Steuerungsarten. Dabei würde man mit Hilfe der Phasenanschnittsteuerung das sanfte Hoch- und Runterfahren der Leistung regulieren, während man mit der Schwingungspaketsteuerung die Anzahl Pakete mit voller Leistung betreibt.\\
Das Ziel dieser Bachelorarbeit ist es, ein solches Verfahren zu analysieren und die wichtigsten Erkenntnisse zu dokumentieren. Dabei sollten zuerst die relevanten Informationen über die gängigen Steuerverfahren für ein- und dreiphasige Wechselspannungssteller herausgefunden werden. Anschliessend sollten analytisch die Strom-Harmonischen Oberschwingungen in Funktion des Zündwinkels der Phasenanschnitt-Steuerung (ein- und dreiphasig) bestimmt werden. Auch die Strom-Harmonische in Funktion des Ein- und Ausschalt-Verhältnisses für Schwingungspaket-Steuerung sind ein wichtiger Bestandteil dieser Arbeit. Die Resultate der Steuerungsverfahren, die man mit den Simulations-Tools Plecs und Matlab aufgebaut hat, sollten mit einem Laboraufbau verglichen werden. Danach betrachtet man ein Verfahren mit sanftem Hoch- und Runterfahren der Leistung und verifiziert das Ganze messtechnisch. Es muss ausserdem darauf geachtet werden, dass die gefundenen Verfahren zwingend die Netzvorschriften(Normen) einhalten. Schlussendlich wird mit einen UniPi-Steuergerät und einer kleinen Software die Steuerverfahren an einer Ohmschen Last und einem Asynchronmotor betrieben. Die wichtigsten Resultate und die Eindrücke sind im nachfolgenden Bericht angegeben.\\
Die vorliegende Projektarbeit gliedert sich in XX Hauptkapitel. Im ersten Teil werden die wichtigsten Grundlagen erläutert. Sie dienen als Anfangskenntnisse der Arbeit und beschreiben die Funktionalität der einzelnen behandelten Verfahren, die aufzutretenen Probleme und die Gegenmassnahmen die man dabei treffen kann. Ausserdem werden die wichtigsten Normen die betrachtet wurden kurz zusammengefasst. Im nächsten Kapitel Simulation sind alle Resultate der simulierten Verfahren, welche man mit Plecs und Matlab dargestellt hat, aufgezeigt. Des weiteren verglich man die gegenseitigen Resultate der Simulationen und konnte sie so verifiziert. Das darauffolgende Kapitel 4 Umsetzung zeigen die Auswertungen des Laboraufbaus mit den verschiedenen angewandten Verbrauchern. Anschliessend Verifiziert man im Kapitel 5 die Resultate des Laboraufbaus mit denen der Plecs-Simulation und betrachtet ob man die angeschauten Normen eingehalten hat. Am Ende folgt in Kapitel 6 die Diskussion zu den verschiedenen Ansteuerungsverfahren und das endgültige Fazit. Des weiteren wird beschrieben inwiefern die gesetzten Ziele erreicht wurden. Bei der Wiederaufnahme des Projektes soll der Bericht eine verständliche Hilfe darstellen.  




















