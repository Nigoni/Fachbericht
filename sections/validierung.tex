\section{Interpretation der Resultate}\label{sec:Interpretation_Resultate}
Bei den Interpretation der Resultate werden noch einmal die unterschiedliche Messungen und deren Auswertung analysiert und wichtige Schlussfolgerungen dokumentiert. Dabei haben das Einhalten der Normen die oberste Priorität. Wenn eine Ansteuerungsart die Normen bei dem Widerstand nicht einhält, muss dies bei der ASM nicht auch der Fall sein. Des Weiteren wird begründet welche Ansteuerungsarten, bei den verschiedenen Lasten, sinnvoll sind. Schlussendlich wird bei den Messungen der höchst beziehungsweise der niedrigste Leistungsfaktor bestimmt.


\subsection{Resultate Widerstand}
Für den Widerstand wurden zuerst die Strommessungen der verschiedenen Ansteuerungsarten durchgeführt. Die Resultate der FFT konnten unmittelbar mit den Stromnormen \todo{Stromnorm Name einfügen} verglichen werden. Dabei war es notwendig, die Ströme auf den maximalen Effektivstrom von \SI{16}{A} hoch zu rechnen. Dazu wurde der höchste Spitzenwert der Messdaten genommen und auf \SI{16}{A} hoch skaliert. Nur so war es möglich die maximalen erlaubten Ströme der Messungen mit den Normen zu vergleichen. Die Norm behandelt die Grenzwerte der harmonischen Oberschwingungsströme. Die Phasenanschnittsteuerungen mit den Winkeln von 60\textdegree \hspace{0.02cm} und 90\textdegree \hspace{0.02cm} erfüllten die Normen nicht. Da bei der Stromnorm nur die Grenzwerte der harmonische Schwingungen aufgelistet sind und diese bei der Schwingungspaketsteuerung nicht auftreten, wurde bei dieser Ansteuerungsart die Ströme nicht gemessen. Das sanfte und harte Auf- und Absteuern hingegen wurde mit den Normen verglichen, da bei diesen Ansteuerungsarten harmonische Schwingungen auftreten. Diese sind jedoch im Verhältnis zur Grundschwingung so klein, dass diese die Grenzwerte der Oberschwingungsströme einhalten. 

Die Norm \todo{Norm} welche die sub- und zwischenharmonischen Schwingungen behandelt, bezieht sich nur auf das \SI{230}{V} System. Deshalb konnten die Messresultate nicht mit dieser Norm verglichen werden. Dadurch wurden für das sanfte und harte Auf- und Absteuern sowie für die Schwingungspaketsteuerung nur die Normen der Oberschwingungsströme betrachtet. Die Auswirkungen der sub- und zwischenharmonischen Schwingungen ist im Kapitel \todo{Norm mit sub und zwisch} beschrieben. Aus der Tabelle \todo{Sub Normen Tabelle} kann jedoch gelesen werden, dass die zugelassenen Grenzwerte welche sich näher bei der Grundschwingung befinden, grösser sind als jene welche weiter weg der Grundschwingung sind. So kann argumentiert werden, dass das sanfte und harte Auf- und Absteuern besser sind als die Schwingungspaketsteuerung, da die Seitenbänder bei der Schwingungspaketsteuerung um die Grundschwingung weiter auseinander sind. So können die Normen nicht auf das \SI{400}{V} System angewendet werden, hingegen kann gesagt werden, je kleiner die Sub- und Zwischenharmonischen sind und je näher sich diese and der Grundfrequenz befinden, desto besser ist die Ansteuerung.

Wenn die Ansteuerungsart die Grenzwerte der Oberschwingungsströme einhalten, wurden die Messwerte der Spannungen mit der Spannungsnorm \todo{Name Spannungsnorm} verglichen. Dabei wurden die 5., 7. und 11. Harmonischen in den Tabellen der Messwerte aufgelistet. Die sonstigen Harmonischen bis zur 23. Ordnung wurden ebenfalls mit der Norm verglichen, jedoch nicht mehr in der Tabelle aufgeführt. Dabei erfüllten das sanfte und harte Auf- und Absteuern ebenfalls die Grenzwerte der Oberspannung.

Der Vergleich der FFT des sanften und harten Auf- und Absteuerns zeigt, dass die 5. Harmonische bei beiden sehr klein ist. Das sanfte Auf- und Absteuern besitzt einen kleineren relativen Wert. Des Weiteren sind die Seitenbänder der Grundschwingung und der Harmonischen näher bei ihren jeweiligen harmonischen Oberschwingungen bei dem sanften Auf- und Absteuern. Welche Auswirkungen diese haben ist nicht genau geklärt es wird vermutet, dass je dünner das Seitenband ist desto weniger Störungen werden ins Netz zurück gespeist.\todo{stimmt das}. Dies hat den Grund, dass je näher die Seitenbänder an ihrer Grundschiwngung sind, desto mehr entsprechen die Seitenbändern den Grundfrequenzen. Wenn zum Beispiel eine Frequenz von \SI{50.35}{Hz} eine relative Amplitude von 62.74\% hat, ist dieses weiter weg von \SI{50}{Hz} als wenn \SI{50.05}{Hz} eine relative Amplitude von 67.76\% hat. \todo{maximale Seitenbänder sind näher bei 50 hz}

Der THD-Wert der Messungen konnte nicht berechnet werden, da diese Berechnungen nur für den stationären Betrieb konzipiert sind. Da keines der Signale welche die Stromnormen erfüllten stationär sind, wurden diese Berechnungen weggelassen. Der Flickerwert wurde ebenfalls nicht gemessen, da keine geeignete Laboreinrichtung und ein Messgerät vorhanden war.

Wenn die Leistungsfaktoren des sanften und harten Auf- und Absteuerns miteinander verglichen werden fällt auf, dass diese fast gleich gross sind. Der kleine Unterschied von 0.0001 kann vernachlässigt werden. Deshalb kann aufgrund des Leistungsfaktors gesagt werden, dass die beiden Ansteuerungsarten gleichwertig sind.\todo{was heisst das jetzt}

Die Sparvariante mit zwei Thyristoren eignet sich nur bedingt für die Ansteuerung einer ohmschen Last. Das harte Auf- und Absteuern erfüllt die Grenzwerte der Oberschwingungsspannung \todo{Name Spannungsnorm einfügen} nicht. Einzig das sanfte Auf- und Absteuern erfüllen die Spannungs- und Stromnorm. Das Problem ist jedoch, dass die Phasen stark unterschiedlich belastet werden. Beim FFT können relative Unterschiede von bis zu 10\% bei den Spannungen und bis zu 28\% bei den Strömen festgestellt werden. 

So kann schlussendlich gesagt werden, dass sich das sanfte Auf- und Absteuern am besten eignet einen Widerstand anzusteuern. Dies Aufgrund des dünnen Seitenbandes bei den harmonischen Oberschwingungen, des hohen Leistungsfaktors und der kleinen Amplitude der auftretenden harmonischen Oberwellen.

\subsection{Resultate ASM}
Da im Voraus nur Strommessungen gemacht wurden, welche für ihre jeweilige Last Sinn ergaben wurden nicht mit allen Ansteuerungsarten die ASM angesteuert. Für die Schwingungspaketsteuerung und das harte Auf- und Abfahren konnten keine Anwendungen gefunden werden, deshalb ergaben diese Messungen keinen Nutzen für die Maschine.

Bei der ASM wurden ebenfalls zuerst die Strommessungen durchgeführt, sodass diese mit den Stromnormen verglichen werden konnten. Wie bei der Strommessung beim Widerstand mussten auch hier die Ströme hoch skaliert werden. Anders als bei den Messungen mit dem Widerstand, erfüllen bei der ASM alle Ansteuerungsarten die Grenzwerte der Stromnormen. Deshalb wurden alle Spannungsmessungen durchgeführt. 

Bei der ASM handelt es sich wie beim Widerstand um ein dreiphasen und nicht ein einphasiges Netz. Deshalb konnten die Normen mit den sub- und zwischenharmonischen Schwingungen nicht mit den Messungen verglichen werden. Da jedoch die Schwingungspaketsteuerung und das harte Auf- und Abfahren bei der ASM nicht verwendet werden, kann gesagt werden dass die Ansteuerungsarten welche mit den Seitenbänder kritisch sind, nicht mehr vorhanden sind. 

Das FFT des Phasenanschnitt 60\textdegree \hspace{0.02cm} zeigt einen fast sauberen Sinus auf, da die harmonischen Oberschwingungen im Verhältnis zur Grundschwingung nur maximal 1.5\% betragen. Ebenfalls treten keine Sub- oder Zwischenharmonische auf. Dies ist auch ersichtlich, wenn das Spannungssignal betrachtet wird. Einzig das nicht saubere steigen und fallen des Sinus resultiert von der nicht ganz schönen Ansteuerung des Thyristorstellers. Beim sanften Auf- und Absteuern treten harmonische Schwingungen auf. Zudem existieren sub- und zwischenharmonische Schwingungen. Im Vergleich der beiden Ansteuerungen kann somit gesagt werden, dass das FFT des Phasenanschnittes mit 60\textdegree \hspace{0.02cm} viel besser ist, als jenes des sanften Auf- und Absteuerns.

Für den Leistungsfaktor müssen nur das sanfte Auf- und Absteuern mit dem Phasenanschnitt mit dem Winkel von 60\textdegree \hspace{0.02cm} verglichen werden. Bei der ASM gilt, je näher der Leistungsfaktor an 0 ist, desto besser. Anders als beim Leistungsfaktor des Widerstandes, gibt es beim Leistungsfaktor der ASM grössere Unterschiede, wobei dieser zwischen dem sanften Auf- und Absteuern und Phasenanschnitt 60\textdegree \hspace{0.02cm} 0.1208 beträgt. Das sanfte Auf- und Abfahren hat einen tieferen Leistungsfaktor als der Phasenanschnitt mit dem Winkel von 60\textdegree. 

Wie auch bei der Widerstandsmessung eignet sich die Sparvariante mit zwei Thyristoren nicht um die ASM anzusteuern. Das sanfte Auf- und Absteuern erfüllt die Grenzwerte der Spannungsnorm nicht und darf deshalb für die Steuerung nicht benutzt werden.

Als Fazit für die verschiedenen Ansteuerungsarten kann gesagt werden, dass sich das sanfte Auf- und Absteuern auch bei der ASM am besten eignet. Zwar zeigte das FFT des Phasenanschnittes mit 60\textdegree \hspace{0.02cm} keine harmonische Oberschwingungen auf, jedoch waren diese beim sanften Auf- und Absteuern nicht sehr gross. Dafür war der Leistungsfaktor des sanften Auf- und Absteuern tiefer als der des Phasenanschnittes mit 60\textdegree.

\subsection{Vergleich der Messungen mit den Simulationen}
Für den Vergleich der Messungen mit den Simulationen gab es grosse Abweichung zwischen den verschiedenen Werten des FFTs. Es stellte sich als schwierig heraus bei komplexeren Ansteuerungsarten wie das sanfte und harte Auf- und Absteuern, die gleichen Zeiten für die Steigung und das Halten auf maximaler Spannung zu erreichen. Ein zusätzliches Problem stellte der Thyristorsteller und die Spannungsverstärkung dar, da diese eine Verzögerung besassen. Zusätzlich werden in der Simulation die Bauteile als ideal angenommen wobei dies bei den Messungen nicht der Fall ist.







