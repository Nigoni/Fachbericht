\section{Interpretation der Resultate}\label{sec:Interpretation_Resultate}
Bei den Interpretationen der Resultate werden noch einmal die unterschiedliche Messungen und deren Auswertung analysiert und wichtige Schlussfolgerungen dokumentiert. Dabei haben das Einhalten der Normen die oberste Priorität. Wenn eine Ansteuerungsart die Normen bei dem Widerstand nicht einhält, muss dies bei der ASM nicht auch der Fall sein. Des Weiteren wird begründet welche Ansteuerungsarten, bei den verschiedenen Lasten, sinnvoll ist. Schlussendlich wird bei den Messungen der höchst beziehungsweise der niedrigste Leistungsfaktor bestimmt.


\subsection{Resultate der Ansteuerungsarten beim Widerstand}
Für den Widerstand wurden zuerst die Strommessungen der verschiedenen Ansteuerungsarten durchgeführt. Die Resultate der FFTs konnten unmittelbar mit den Grenzwerten der Oberschwingungsströme \ref{sec:Stromnormen} verglichen werden. Dabei wurden, die Ströme auf den maximalen Effektivstrom von \SI{16}{A} hoch gerechnet. Dazu wurde der höchste Spitzenwert der Messdaten genommen und auf \SI{16}{A} hoch skaliert. Nur so war es möglich die maximalen erlaubten Ströme der Messungen mit den Normen zu vergleichen. Die Phasenanschnittsteuerungen mit den Winkeln von 60\textdegree \hspace{0.02cm} und 90\textdegree \hspace{0.02cm} erfüllten die Normen nicht. Da bei der Oberschwingungsströme nur die Grenzwerte der harmonische Schwingungen aufgelistet sind und diese bei der Schwingungspaketsteuerung nicht auftreten, wurde bei dieser Ansteuerungsart die Ströme nicht gemessen. Das sanfte und harte Auf- und Absteuern hingegen verglich man mit den Normen \ref{sec:Stromnormen}, da bei diesen Ansteuerungsarten harmonische Oberschwingungsströme auftreten. Diese sind jedoch im Verhältnis zur Grundschwingung so klein, dass sie die Grenzwerte einhalten.\\ 

Wenn die Ansteuerungsart die Grenzwerte der Oberschwingungsströme einhalten, wurden die Messwerte der Spannungen mit den Grenzwerten der harmonischen Oberschwingungsspannung, die in der Norm 61000-2-2 \ref{sec:Spannungsnormen} behandelt wurden, verglichen. Dabei sind die 5., 7. und 11. Harmonischen in den Tabellen der Messwerte aufgelistet. Die sonstigen harmonischen Oberschwingungen, bis zur 23. Ordnung, wurden ebenfalls mit der Norm verglichen, jedoch nicht mehr in die jeweiligen Tabellen eingeführt. Dabei erfüllten das sanfte und harte Auf- und Absteuern ebenfalls die Grenzwerte der Oberschwingungsspannung.\\

Die Norm 61000-2-2 enthält ausserdem die Grenzwerte, welche die Spannungen der sub- und zwischenharmonischen Schwingungen einhalten müssen. Sie bezieht sich jedoch nur auf die \SI{230}{V}-Systeme. Deshalb konnten die Messresultate nicht mit dieser Norm verglichen werden. Die Auswirkungenen der sub- und zwischenharmonischen Schwingungen sind im Kapitel \ref{sec:Spannungsnormen} beschrieben. Aus der Tabelle \ref{tab:kompatibilitätsstufen} wurde jedoch erkannt, dass die zugelassenen Grenzwerte welche sich näher bei der Grundschwingung befinden, ein grösser Peak-Wert haben dürfen, als jene die weiter weg von der Grundschwingung sind. So kann argumentiert werden, dass beim sanften und harten Auf- und Absteuern die Seitenbänder um die Grundschwingung näher beieinander sind als bei der Schwingungspaketsteuerung. Welche Auswirkungen der Abstand der Seitenbänder hat, konnte nicht geklärt werden. Es wird vermutet, dass je näher die Bänder an den jeweiligen harmonischen Ordnung sind, desto weniger Störungen werden ins Netz zurück gespeist. Somit würden die sanfte- und harte Ansteuerung einen saubereren Sinus heraus geben als die Schwingungspaketsteuerungen. Vergleicht man die Seitenbänder der erst genannten Verfahren miteinander erkennt man, dass die Frequenzen, bei denen die sub- und zwischenharmoisch Schwingungen auftreten, beim der sanft-Ansteuerung näher beieinander sind.\\

Der THD-Wert der Messungen ist nicht bestimmt worden, da diese Berechnungen nur für den stationären Betrieb konzipiert sind. Da keines der Signale welche die Stromnormen erfüllten stationär sind, wurden diese Berechnungen weggelassen. Der Flickerwert wurde ebenfalls nicht gemessen, da keine geeignete Laboreinrichtung und kein Messgerät vorhanden war.\\

Wenn die Leistungsfaktoren des sanften und harten Auf- und Absteuerns miteinander verglichen werden fällt auf, dass diese fast gleich gross sind. Der kleine Unterschied von 0.0001 kann vernachlässigt werden. Deshalb kann aufgrund des Leistungsfaktors gesagt werden, dass die beiden Ansteuerungsarten gleichwertig sind.\\

Die Sparvariante mit zwei Thyristoren eignet sich nur bedingt für die Ansteuerung einer ohmschen Last. Das harte Auf- und Absteuern erfüllt die Grenzwerte der Oberschwingungsspannung \ref{sec:Spannungsnormen} nicht. Einzig das sanfte Auf- und Absteuern erfüllen die Spannungs- und Stromnorm. Das Problem ist jedoch, dass die Phasen stark unterschiedlich belastet werden. Beim FFT können relative Unterschiede bei den Spannungen von bis zu 10\% und bei den Strömen von bis zu 28\% festgestellt werden.\\

Schlussendlich erkannte man, dass sich das sanfte Auf- und Absteuern am besten eignet um einen Widerstand anzusteuern. Dies Aufgrund der dünnen Seitenbänder bei den harmonischen Oberschwingungen, des hohen Leistungsfaktors und der kleinen Amplitude der auftretenden harmonischen Oberwellen.

\subsection{Resultate der Ansteuerungsarten bei der ASM}
Da im Voraus die Strommessungen vollzogen wurden, welche für ihre jeweilige Last Sinn ergeben, wurden nicht mit allen Ansteuerungsarten die ASM angesteuert. Für die Schwingungspaketsteuerung und das harte Auf- und Abfahren der Leistung ist keine Anwendungen für die Maschine bekannt. Deshalb verzichtete man auf die Messungen dieser Steuerungsarten.\\


Bei der Asynchronmaschine wurden zuerst die Strommessungen der Phasenanschnittsteuerung mit den bekannten Winkeln von 60\textdegree und 90\textdegree \hspace{0.02cm} sowie das sanfte Auf- und Asteuern des Stromes, durchgeführt. Die Werte des Effektivstromes wurde wieder mit einem Faktor auf die \SI{16}{A} hoch skaliert, um den maximalen erlaubten Stromstromwerte zu erhalten. So konnten die Grenzwerte der Oberschwingungsströme mit den dazugehörigen Normen \ref{sec:Normen} verglichen werden. Anders als bei dem Widerstand, erfüllen alle untersuchten Ansteuerungsarten die Grenzwerte der Normen. Deshalb konnte man in einem zweiten Messverfahren die Oberschwingungsspannungen untersuchen.\\

Bei der ASM handelt es sich wie auch beim Widerstand um ein \SI{400}{V}- und nicht ein \SI{230}{V}-System. Deshalb konnten die sub- und zwischenharmonischen Schwingungen der Messungen nicht mit den Normen\ref{sec:Spannungsnormen} verglichen werden.
Da jedoch die Schwingungspaketsteuerung und das harte Auf- und Abfsteuern der ASM nicht verwendet wird, schloss man daraus, dass die Ansteuerungsarten welche kritische Seitenbänder hervorruft, nicht mehr vorhanden sind. 

Das FFT des Phasenanschnitt 60\textdegree \hspace{0.02cm} zeigt bei allen drei Phasen, einen fast perfekten Sinus auf.
Die harmonischen Oberschwingungen im Verhältnis zur Grundschwingung betragen desshalb nur maximal 1.5\%. Sub- und Zwischenharmonische sind ebenfalls kein vorhanden. Dies erkennt man auch nach betrachten des Spannungssignals.
Der Phasenanschnitt mit 90\textdegree \hspace{0.02cm} ist für die Ansteuerung der ASM nicht geeignet, da die harmonischen Oberschwingungen die Grenzwerte der Norm \ref{sec:Spannungsnormen} nicht einhalten. 
Beim sanften Auf- und Absteuern sind harmonische Schwingungen zu erkennen. Sie halten jedoch die Grenzwerte ein. Zudem existieren sub- und zwischenharmonische Schwingungen. Im Vergleich zur Ansteuerung mit dem Phasenanschnitt von 60\textdegree \hspace{0.02cm} sind die Verhältnisse zur Grundschwingung grösser.\\ 

Für den Leistungsfaktor verglich man nur noch das sanfte Auf- und Absteuern mit der Phasenanschnittsteuerung mit einem Winkel von 60\textdegree \hspace{0.02cm}. Bei der ASM gilt, je näher der Leistungsfaktor bei 0 ist, desto niedriger sind die Ummagnetisierungs- und Eisenverluste. Anders als beim Faktor des Widerstandes, gibt es beim Leistungsfaktor der ASM grössere Unterschiede, wobei dieser zwischen dem sanften Auf- und Absteuern und Phasenanschnitt 60\textdegree \hspace{0.02cm} 0.1208 beträgt. Das sanfte Auf- und Abfahren hat einen tieferen Leistungsfaktor als der Phasenanschnitt mit dem Winkel von 60\textdegree. \\

Wie auch bei der Widerstandsmessung wurde bei der ASM eine Sparvariante mit zwei Thyristoren untersucht. Sie haltet jedoch die Grenzwerte der Oberschwingungsströme nicht ein. Die Variante wurde deshalb nicht mehr aufgezeigt.\\

Als Fazit für die verschiedenen Ansteuerungsarten erkannte man, dass sich das sanfte Auf- und Absteuern auch bei der ASM am besten eignet. Zwar zeigte das FFT des Phasenanschnittes mit 60\textdegree \hspace{0.02cm} keine harmonische Oberschwingungen, jedoch waren sie beim sanften Auf- und Absteuern in Rahmen der erlaubten Grenzwerte der Normen. Der Leistungsfaktor des sanften Auf- und Absteuern ist tiefer als der des Phasenanschnittes mit 60\textdegree \hspace{0.02cm} und deshalb besser.

\subsection{Vergleich der Messungen mit den Simulationen}
Für den Vergleich der Messungen mit den Simulationen gab es grosse Abweichung zwischen den verschiedenen Werten der FFTs. Es stellte sich als schwierig heraus bei komplexeren Ansteuerungsarten wie das sanfte und harte Auf- und Absteuern, die gleichen Zeiten für die Steigung und das Halten auf maximaler Spannung zu erreichen. Ein zusätzliches Problem stellte der Thyristorsteller und die Spannungsverstärkung dar, da diese eine Verzögerung besassen. Zusätzlich werden in der Simulation die Bauteile als ideal angenommen wobei dies bei den Messungen nicht der Fall ist. Erste Einblicke in das Simulation-Tool Plecs und das Verständnis des Verhalten von verschiedenen Ansteuerungsarten konnten trotzdem gewonnen werden. 







