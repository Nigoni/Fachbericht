\begin{appendix} %Anhang
\section{Matlab-Berechnungen}
\subsection{Leistungsfaktor}
\lstinputlisting{appendix/code/Leistungsfaktor.m}

\newpage
\subsection{Arduino-Programm}
\begin{lstlisting}[language=Arduino]
const int Switch_PO = 3;         // Output fuer den EIN- und AUSSCHALTER
const int Controll_P = 2;   // Output fuer Steuersignal
const int Switch_PI = 5; // Input fuer den EIN- und AUSSCHALTER
int buttonState = 0;                 // Zustand Schalter auf 0
int anzahl_Schwinwungspakete = 5;        // Anzahl Schwingungspakete

void setup() {
pinMode (Switch_PO, OUTPUT);        // Schalter auf Output
pinMode (Controll_P, OUTPUT);              // PWM auf Output
pinMode (Switch_PI, INPUT_PULLUP);      // Schalter Eingang
}

void loop() {
buttonState =! digitalRead(Switch_PI);      // Da Pullup wird das Signal negiert
if(buttonState == HIGH){          // if-Schleife falls Zustand Schalter auf EIN
for(int z=0; z<10; z++){             // for-Schleife fuer Schwingungspaketsteuerung
if(z<anzahl_Schwinwungspakete){                 // Anzahl Schwingungen EIN
for(int i=0; i<255; i++){                     // Hochfahren mit PWM
analogWrite(Controll_P, i);  // Die Variable wird auf den Steuersignalausgang geschrieben
delay(5);// Kurze Zeitverzoegerung da sonst das PWM zu schnell fuer ein Multimeter hochfaehrt       
}
delay(200);     // Warten waehrend volle Leistung
for(int i=255; i>0; i--){                     // Runterfahren mit PWM
analogWrite(Controll_P, i);                 // Die Variable wird auf den Steuersignalausgang geschrieben
delay(5); // Kurze Zeitverzoegerung da sonst das PWM zu schnell fuer ein Multimeter runterfaehrt
}
delay(100);   // Warten waehrend keiner Leistung
}else{
digitalWrite(Controll_P, LOW); // Ausschalten fuer Schwingungspakete keine Leistung
delay((10-anzahl_Schwinwungspakete)*480);     // Verzoegerung bis wieder eingeschaltet werden soll
			}
}
}
}





\end{lstlisting}

\newpage
\section{Vergleich der Resultate von Plecs und Matlab}\label{sec:Vergleich_der_Resultate}

\begin{figure}[ht!]
	\centering
	\subfloat[][]{\includegraphics[width=0.4\linewidth]{eingangssignal_30.png}\label{fig:eingangssignal_30}}\qquad
	\subfloat[][]{\includegraphics[width=0.4\linewidth]{A_PH_30.png}\label{fig:A_PH_30.}}
	\caption{Phasenanschnittsteuerung mit 30\textdegree (a) Eingangssignal (b) Amplituden- und Phasenspektrum}
	\label{fig:Phasenanschnittsteuerung_mit_30}
\end{figure}

\begin{figure}[ht!]
	\centering
	\subfloat[][]{\includegraphics[width=0.4\linewidth]{eingangssignal_45.png}\label{fig:eingangssignal_45}}\qquad
	\subfloat[][]{\includegraphics[width=0.4\linewidth]{A_PH_45.png}\label{fig:A_PH_45}}
	\caption{Phasenanschnittsteuerung mit 45\textdegree (a) Eingangssignal (b) Amplituden- und Phasenspektrum}
	\label{fig:Phasenanschnittsteuerung_mit_45}
\end{figure}

\begin{figure}[ht!]
	\centering
	\subfloat[][]{\includegraphics[width=0.4\linewidth]{eingangssignal_120.png}\label{fig:eingangssignal_120}}\qquad
	\subfloat[][]{\includegraphics[width=0.4\linewidth]{A_PH_120.png}\label{fig:A_PH_120}}
	\caption{Phasenanschnittsteuerung mit 120\textdegree (a) Eingangssignal (b) Amplituden- und Phasenspektrum}
	\label{fig:Phasenanschnittsteuerung_mit_120}
\end{figure}

\newpage

\begin{figure}[ht!]
	\centering
	\subfloat[][]{\includegraphics[width=0.4\linewidth]{Schwingungspaket_0_2.png}\label{fig:Schwingungspaket_0_2}}\qquad
	\subfloat[][]{\includegraphics[width=0.4\linewidth]{Oberwellen_0_2.png}\label{fig:Oberwellen_0_2}}
	\caption{Schwingungspaket mit Duty Cycle 0.2 (a) Eingangssignal (b) Amplituden- und Phasenspektrum}
	\label{fig:Schwingungspaketsteuerung_mit_duty_cycle_0_2}
\end{figure}


\begin{figure}[ht!]
	\centering
	\subfloat[][]{\includegraphics[width=0.4\linewidth]{Schwingungspaket_0_4.png}\label{fig:Schwingungspaket_0_4}}\qquad
	\subfloat[][]{\includegraphics[width=0.4\linewidth]{Oberwellen_0_4.png}\label{fig:Oberwellen_0_4}}
	\caption{Schwingungspaket mit Duty Cycle 0.4 (a) Eingangssignal (b) Amplituden- und Phasenspektrum}
	\label{fig:Schwingungspaketsteuerung_mit_duty_cycle_0_4}
\end{figure}

\begin{figure}[ht!]
	\centering
	\subfloat[][]{\includegraphics[width=0.32\linewidth]{plecs_phasenanschnitt_pi_6_funktion.png}\label{fig:plecs_phasenanschnitt_pi_6_funktion}}\qquad
	\subfloat[][]{\includegraphics[width=0.32\linewidth]{plecs_phasenanschnitt_pi_6.png}\label{fig:plecs_phasenanschnitt_pi_6}}
	\caption{Phasenanschnitt mit 30\textdegree simuliert mit Plecs (a) Eingangssignal (b) Amplituden- und Phasenspektrum}
	\label{fig:Plecs_mit_phasenanschnitt_30}
\end{figure}

\newpage

\begin{figure}[ht!]
	\centering
	\subfloat[][]{\includegraphics[width=0.32\linewidth]{plecs_phasenanschnitt_pi_4_funktion.png}\label{fig:plecs_phasenanschnitt_pi_4_funktion}}\qquad
	\subfloat[][]{\includegraphics[width=0.32\linewidth]{plecs_phasenanschnitt_pi_4.png}\label{fig:plecs_phasenanschnitt_pi_4}}
	\caption{Phasenanschnitt mit 45\textdegree simuliert mit Plecs (a) Eingangssignal (b) Amplituden- und Phasenspektrum}
	\label{fig:Plecs_mit_phasenanschnitt_45}
\end{figure}


\begin{figure}[ht!]
	\centering
	\subfloat[][]{\includegraphics[width=0.32\linewidth]{plecs_phasenanschnitt_120_funktion.png}\label{fig:plecs_phasenanschnitt_120_funktion}}\qquad
	\subfloat[][]{\includegraphics[width=0.32\linewidth]{plecs_phasenanschnitt_120.png}\label{fig:plecs_phasenanschnitt_120}}
	\caption{Phasenanschnitt mit 120\textdegree simuliert mit Plecs (a) Eingangssignal (b) Amplituden- und Phasenspektrum}
	\label{fig:Plecs_mit_phasenanschnitt_120}
\end{figure}


\begin{figure}[ht!]
	\centering
	\subfloat[][]{\includegraphics[width=0.32\linewidth]{plecs_schwingungspacket_0_2_schwingungen.PNG}\label{fig:plecs_schwingungspacket_0_2_schwingungen}}\qquad
	\subfloat[][]{\includegraphics[width=0.32\linewidth]{plecs_schwingungspacket_0_2_1000.PNG}\label{fig:plecs_schwingungspacket_0_2_1000}}
	\caption{Schwingungspaketsteuerung mit Duty Cycle 0.2 simuliert mit Plecs (a) Eingangssignal (b) Amplitudenspektrum}
	\label{fig:Schwingungspaketsteuerung_mit_duty_cycle_0_2 simuliert_mit_Plecs}
\end{figure}

\newpage

\begin{figure}[ht!]
	\centering
	\subfloat[][]{\includegraphics[width=0.32\linewidth]{plecs_schwingungspacket_0_4_schwingungen.PNG}\label{fig:plecs_schwingungspacket_0_4_schwingungen}}\qquad
	\subfloat[][]{\includegraphics[width=0.32\linewidth]{plecs_schwingungspacket_0_2_1000.PNG}\label{fig:plecs_schwingungspacket_0_4_1000}}
	\caption{Schwingungspaketsteuerung mit Duty Cycle 0.4 simuliert mit Plecs (a) Eingangssignal (b) Amplitudenspektrum}
	\label{fig:Schwingungspaketsteuerung_mit_duty_cycle_0_4 simuliert_mit_Plecs}
\end{figure}


\begin{figure}[ht!]
	\centering
	\includegraphics[scale=0.55]{Vergleich_Amplitudenspektrum_mit_Phasenwinkel_90.png}	
	\caption{Vergleich der Amplitudenspektrum mit Phasenwinkel von 90\textdegree}
	\label{fig:Vergleich_der_Amplitudenspektrum_mit Phasenwinkel_von_90}
\end{figure}

\begin{figure}[ht!]
	\centering
	\includegraphics[scale=0.55]{Vergleich_einphasiges_Schwingungspaket_mit_duty_cycle_von_0_5.png}	
	\caption{Vergleich des Schwingungspaket mit Duty Cycle von 0.5}
	\label{fig:Vergleich des Schwingungspaket mit Duty Cycle von 0.5}
\end{figure}

\newpage

\begin{figure}[ht!]
	\centering
	\includegraphics[scale=0.55]{Vergleich_einphasiges_Schwingungspaket_mit_duty_cycle_von_0_8.png}	
	\caption{Vergleich des Schwingungspaket mit Duty Cycle von 0.8}
	\label{fig:Vergleich des Schwingungspaket mit Duty Cycle von 0.8}
\end{figure}

\begin{figure}[ht!]
	\centering
	\includegraphics[scale=0.55]{Vergleich_absolut_logarithmic_duty_cycle_von_0_5_mit_legende.PNG}	
	\caption{Vergleich des Schwingungspaket mit Duty Cycle von 0.8}
	\label{fig:Vergleich_absolut_logarithmic_duty_cycle_von_0_5_mit_legende}
\end{figure}

\begin{figure}[ht!]
	\centering
	\includegraphics[scale=0.55]{Vergleich_absolut_logarithmic_duty_cycle_von_0_8_mit_legende.PNG}	
	\caption{Vergleich des Schwingungspaket mit Duty Cycle von 0.8}
	\label{fig:Vergleich_absolut_logarithmic_duty_cycle_von_0_8_mit_legende}
\end{figure}

\newpage
\subsection{Messungen Spannungen Widerstand}\label{sec:Mess_Spannung_Widerstand}
\subsubsection*{Phasenanschnitt 60\textdegree}

\begin{figure}[ht!]
	\centering
	\includegraphics[width=\textwidth]{Messung_Widerstand_Phas_60grad.png}	
	\caption{Messung mit Phasenanschnitt 60\textdegree}\label{fig:Mess_Phas_60}
\end{figure}
\newpage
\subsubsection*{Phasenanschnitt 90\textdegree}
\begin{figure}[ht!]
	\centering
	\includegraphics[width=\textwidth]{Messung_Widerstand_Phas_90grad.png}	
	\caption{Messung mit Phasenanschnitt 90\textdegree}\label{fig:Mess_Phas_90}
\end{figure}

\newpage
\subsection{Messungen Spannungen ASM}\label{sec:Mess_Spannung_ASM}
\subsubsection*{Schwingungspaket 50\%}
\begin{figure}[ht!]
	\centering
	\includegraphics[width=\textwidth]{Messung_ASM_Schwing_0_5.png}	
	\caption{Messung mit Schwingungspaket 50\%}\label{fig:Mess_ASM_Schwing_0_5}
\end{figure}

\newpage
\subsubsection*{Schwingungspaket 80\%}
\begin{figure}[ht!]
	\centering
	\includegraphics[width=\textwidth]{Messung_ASM_Schwing_0_8.png}	
	\caption{Messung mit Schwingungspaket 80\%}\label{fig:Mess_ASM_Schwing_0_8}
\end{figure}

\newpage
\subsubsection*{Auf- und Absteuern}
\begin{figure}[ht!]
	\centering
	\includegraphics[width=\textwidth]{Messung_ASM_Sanft.png}	
	\caption{Messung mit Sanft Anlasser}\label{fig:Mess_ASM_Sanft}
\end{figure}

\newpage
\subsection{Vergleich Messungen Widerstand mit Simulation}
\subsubsection*{Schwingungspaketsteuerung 50\%} \label{sec:Vergleich_Mess_Sim_Schwing_50}

\begin{figure}[ht!]
	\begin{minipage}[b]{0.59\textwidth}
		\centering
		\includegraphics[width=\textwidth]{Vergleich_Mess_Sim_Schwing_50.png}	
		\caption{Vergleich Messung und Simulation mit Schwingungspaketsteuerung 50\%}\label{fig:Vergleich_Schwing_50}
	\end{minipage}
	%
	\begin{minipage}[b]{0.4\textwidth}
		\centering
		\begin{tabular}{|l|l|l|}
		\hline
		\begin{tabular}[c]{@{}l@{}}Fre-\\ quenz\\ {[}Hz{]}\end{tabular} & \begin{tabular}[c]{@{}l@{}}Simu-\\ lation\\ {[}V{]}\end{tabular} & \begin{tabular}[c]{@{}l@{}}Mes\\ sung\\ {[}V{]}\end{tabular} \\ \hline
		46                                                              & 1.676                                                            & 14.735                                                       \\ \hline
		47                                                              & 35.352                                                           & 28                                                           \\ \hline
		48                                                              & 1.667                                                            & 26.376                                                       \\ \hline
		49                                                              & 102.412                                                          & 95.6                                                         \\ \hline
		50                                                              & 161.597                                                          & 149.92                                                       \\ \hline
		51                                                              & 101.858                                                          & 99.8                                                         \\ \hline
		52                                                              & 1.648                                                            & 25.134                                                       \\ \hline
		53                                                              & 33.293                                                           & 20.6                                                         \\ \hline
		\end{tabular}
		\caption{Vergleich Messung und Simulation mit Schwingungspaketsteuerung 50\%}\label{tab:Vergleich_Schwing_50}
	\end{minipage}
\end{figure}


\subsubsection*{Schwingungspaketsteuerung 80\%} \label{sec:Vergleich_Mess_Sim_Schwing_80}
\begin{figure}[ht!]
	\begin{minipage}[b]{0.59\textwidth}
		\centering
		\includegraphics[width=\textwidth]{Vergleich_Mess_Sim_Schwing_80.png}	
		\caption{Vergleich Messung und Simulation mit Schwingungspaketsteuerung 80\%}\label{fig:Vergleich_Schwing_80}
	\end{minipage}
	%
	\begin{minipage}[b]{0.4\textwidth}
		\centering
			\begin{tabular}{|l|l|l|}
			\hline
			\begin{tabular}[c]{@{}l@{}}Fre-\\ quenz\\ {[}Hz{]}\end{tabular} & \begin{tabular}[c]{@{}l@{}}Simu-\\ lation\\ {[}V{]}\end{tabular} & \begin{tabular}[c]{@{}l@{}}Mes\\ sung\\ {[}V{]}\end{tabular} \\ \hline
			46                                                              & 16.335                                                           & 20.173                                                       \\ \hline
			47                                                              & 32.649                                                           & 28.26                                                        \\ \hline
			48                                                              & 47.944                                                           & 40.576                                                       \\ \hline
			49                                                              & 61.08                                                            & 62.694                                                       \\ \hline
			50                                                              & 260.212                                                          & 265.98                                                       \\ \hline
			51                                                              & 59.54                                                            & 65.7                                                         \\ \hline
			52                                                              & 47.781                                                           & 43.812                                                       \\ \hline
			53                                                              & 31.664                                                           & 21.939                                                       \\ \hline
		\end{tabular}
		\caption{Vergleich Messung und Simulation mit Schwingungspaketsteuerung 80\%}\label{tab:Vergleich_Schwing_80}
	\end{minipage}
\end{figure}



\newpage
\subsubsection*{Hartes Auf- und Absteuern}\label{sec:Vergleich_Mess_Sim_hart_AufAb}
\begin{figure}[ht!]
	\begin{minipage}[b]{0.59\textwidth}
		\centering
		\includegraphics[width=\textwidth]{Vergleich_Mess_Sim_hart_AufAb.png}	
		\caption{Vergleich Messung und Simulation mit hartem Auf- und Absteuern}\label{fig:Vergleich_hart_AufAb}
	\end{minipage}
	%
	\begin{minipage}[b]{0.4\textwidth}
		\centering
		\begin{tabular}{|l|l|l|}
			\hline
			\begin{tabular}[c]{@{}l@{}}Fre-\\ quenz\\ {[}Hz{]}\end{tabular} & \begin{tabular}[c]{@{}l@{}}Simu-\\ lation\\ {[}V{]}\end{tabular} & \begin{tabular}[c]{@{}l@{}}Mes-\\ sung\\ {[}V{]}\end{tabular} \\ \hline
			49.8               & 66.953             & 18.522          \\ \hline
			49.85              & 24.870             & 26.576          \\ \hline
			49.9               & 174.378            & 29.507          \\ \hline
			49.95              & 314.127            & 91.266          \\ \hline
			50                 & 370.962            & 172.241         \\ \hline
			50.05              & 314.051            & 116.719         \\ \hline
			50.1               & 174.266            & 28.629          \\ \hline
			50.15              & 24.871             & 30.076          \\ \hline
			50.2               & 66.83              & 18.72           \\ \hline
			249.6              & 0.188              & 8.158           \\ \hline
			250                & 0.4967             & 1.158           \\ \hline
			250.4              & 0.445              & 7.466           \\ \hline
		\end{tabular}
		\caption{Vergleich Messung und Simulation mit hartem Auf- und Absteuern}\label{tab:Vergleich_hart_AufAb}
	\end{minipage}
\end{figure}


\subsubsection*{Sanftes Auf- und Absteuern}\label{sec:Vergleich_Mess_Sim_sanft_AufAb}
\begin{figure}[ht!]
	\begin{minipage}[b]{0.59\textwidth}
		\centering
		\includegraphics[width=\textwidth]{Vergleich_Mess_Sim_sanft_AufAb.png}	
		\caption{Vergleich Messung und Simulation mit sanftem Auf- und Absteuern}\label{fig:Vergleich_sanft_AufAb}
	\end{minipage}
%
	\begin{minipage}[b]{0.4\textwidth}
	\centering
		\begin{tabular}{|l|l|l|}
			\hline
			\begin{tabular}[c]{@{}l@{}}Fre-\\ quenz\\ {[}Hz{]}\end{tabular} & \begin{tabular}[c]{@{}l@{}}Simu-\\ lation\\ {[}V{]}\end{tabular} & \begin{tabular}[c]{@{}l@{}}Mes-\\ sung\\ {[}V{]}\end{tabular} \\ \hline
			49.8               & 66.953             & 18.522          \\ \hline
			49.85              & 24.870             & 26.576          \\ \hline
			49.9               & 174.378            & 29.507          \\ \hline
			49.95              & 314.127            & 91.266          \\ \hline
			50                 & 370.962            & 172.241         \\ \hline
			50.05              & 314.051            & 116.719         \\ \hline
			50.1               & 174.266            & 28.629          \\ \hline
			50.15              & 24.871             & 30.076          \\ \hline
			50.2               & 66.83              & 18.72           \\ \hline
			249.6              & 0.188              & 8.158           \\ \hline
			250                & 0.4967             & 1.158           \\ \hline
			250.4              & 0.445              & 7.466           \\ \hline
		\end{tabular}
	\caption{Vergleich Messung und Simulation mit sanftem Auf- und Absteuern}\label{tab:Vergleich_sanft_AufAb}
	\end{minipage}
\end{figure}


%\begin{table}[]
%	\begin{tabular}{|l|l|l|}
%		\hline
%		\begin{tabular}[c]{@{}l@{}}Fre-\\ quenz\\ {[}Hz{]}\end{tabular} & \begin{tabular}[c]{@{}l@{}}Simu-\\ lation\\ {[}V{]}\end{tabular} & \begin{tabular}[c]{@{}l@{}}Mes\\ sung\\ {[}V{]}\end{tabular} \\ \hline
%		49.8                                                            & 66.953                                                           & 18.522                                                       \\ \hline
%		49.85                                                           & 24.870                                                           & 26.576                                                       \\ \hline
%		49.9                                                            & 174.378                                                          & 29.507                                                       \\ \hline
%		49.95                                                           & 314.127                                                          & 91.266                                                       \\ \hline
%		50                                                              & 370.962                                                          & 172.241                                                      \\ \hline
%		50.05                                                           & 314.051                                                          & 116.719                                                      \\ \hline
%		50.1                                                            & 174.266                                                          & 28.629                                                       \\ \hline
%		50.15                                                           & 24.871                                                           & 30.076                                                       \\ \hline
%		50.2                                                            & 66.83                                                            & 18.72                                                        \\ \hline
%		249.6                                                           & 0.188                                                            & 8.158                                                        \\ \hline
%		250                                                             & 0.4967                                                           & 1.158                                                        \\ \hline
%		250.4                                                           & 0.445                                                            & 7.466                                                        \\ \hline
%	\end{tabular}
%\end{table}

\newpage
\subsection{Spar-Variante für den Widerstand mit zwei Thyristoren}
\subsubsection*{Phasenanschnitt 60\textdegree}

\begin{figure}[ht!]
	\centering
	\includegraphics[width=\textwidth]{Mess_2Thyristoren_Widerstand_Phas_60.png}	
	\caption{Messung mit Phasenanschnitt 60\textdegree \hspace{0.02cm} und zwei Thyristoren}\label{fig:Mess_2Thyristoren_Phas_60grad}
\end{figure}

\newpage
\subsubsection*{Phasenanschnitt 90\textdegree}

\begin{figure}[ht!]
	\centering
	\includegraphics[width=\textwidth]{Mess_2Thyristoren_Widerstand_Phas_90.png}	
	\caption{Messung mit Phasenanschnitt 90\textdegree \hspace{0.02cm} und zwei Thyristoren}\label{fig:Mess_2Thyristoren_Phas_90grad}
\end{figure}

\newpage
\subsubsection*{Schwingungspaketsteuerung 50\%}

\begin{figure}[ht!]
	\centering
	\includegraphics[width=\textwidth]{Mess_2Thyristoren_Widerstand_Schwing_05.png}	
	\caption{Messung mit Schwingungspaket 50\% und zwei Thyristoren}\label{fig:Mess_2Thyristoren_Schwing_50}
\end{figure}

\newpage
\subsubsection*{Schwingungspaketsteuerung 80\%}

\begin{figure}[ht!]
	\centering
	\includegraphics[width=\textwidth]{Mess_2Thyristoren_Widerstand_Schwing_08.png}	
	\caption{Messung mit Schwingungspaket 50\% und zwei Thyristoren}\label{fig:Mess_2Thyristoren_Schwing_80}	
\end{figure}


\newpage
\subsection{Spar-Variante für den Widerstand mit einem Thyristor}
\subsubsection*{Phasenanschnitt 60\textdegree}

\begin{figure}[ht]
	\centering
	\includegraphics[width=\textwidth]{Mess_1Thyristor_Widerstand_Phas60.png}	
	\caption{Messung mit Phasenanschnitt 60\textdegree \hspace{0.02cm} und einem Thyristoren}\label{fig:Mess_1Thyristor_Phas_60grad}
\end{figure}

\newpage
\subsubsection*{Phasenanschnitt 90\textdegree}

\begin{figure}[ht]
	\centering
	\includegraphics[width=\textwidth]{Mess_1Thyristor_Widerstand_Phas90.png}	
	\caption{Messung mit Phasenanschnitt 90\textdegree \hspace{0.02cm} und einem Thyristoren}\label{fig:Mess_1Thyristor_Phas_90grad}
\end{figure}

\newpage
\subsubsection*{Schwingungspaketsteuerung 50\%}

\begin{figure}[ht]
	\centering
	\includegraphics[width=\textwidth]{Mess_1Thyristor_Widerstand_Schwing_05.png}	
	\caption{Messung mit Schwingungspaket 50\% und einem Thyristoren}\label{fig:Mess_1Thyristor_Schwing_50}
\end{figure}

\newpage
\subsubsection*{Schwingungspaketsteuerung 80\%}

\begin{figure}[ht]
	\centering
	\includegraphics[width=\textwidth]{Mess_1Thyristor_Widerstand_Schwing_08.png}	
	\caption{Messung mit Schwingungspaket 50\% und einem Thyristoren}\label{fig:Mess_1Thyristoren_Schwing_80}	
\end{figure}

\newpage
\subsubsection*{Auf- und Absteuern}

\begin{figure}[ht]
	\centering
	\includegraphics[width=\textwidth]{Mess_1Thyristor_Widerstand_AufAb.png}	
	\caption{Messung mit Auf- und Absteuern und einem Thyristoren}\label{Mess_1Thyristoren_Widerstand_AufAbFahren}	
\end{figure}

\newpage
\subsubsection*{Langsames Auf- und Absteuern}

\begin{figure}[ht]
	\centering
	\includegraphics[width=\textwidth]{Mess_1Thyristor_Widerstand_AufAb_langsam.png}	
	\caption{Messung mit dem langsamen Auf- und Absteuern und einem Thyristoren}\label{Mess_1Thyristoren_Widerstand_AufAbFahren_langsam}	
\end{figure}

\end{appendix}


