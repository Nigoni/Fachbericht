\section{Schlusswort}
Das Ziel der vorliegenden Bachelorarbeit war es, eine Kombination von zwei geläufigen Ansteuerungsverfahren zu entwickeln, welche störende Oberschwingungen minimiert. Anhand eines Laboraufbaus sollte die gefunden Steuerung mit den Grenzwerten der Normen geprüft werden.\\ 
Die Ergebnisse der beiden Verfahren, Phasenanschnitt- und Schwingungspaketsteuerung, konnten analytisch auf verlustbehaftete Oberwellen untersucht werden. Dabei wurde erkannt, dass je grösser der Winkel des Phasenanschnittes gewählt wird, desto gösser werden die Amplitudenwerte der harmonischen Oberschwingungen. Bei der Schwingungspaketsteuerung treten, bei kleinerem Duty Cycle, mehr Sub- und Zwischenharmonische auf. Mithilfe der Simulations-Tools, Matlab und Plecs, konnten die einphasige analytische Untersuchung, der beiden Steuerverfahren, ausgeführt und verifiziert werden. Es zeigte sich, dass beide Tools die gleichen Werte des Amplitudenspektrums erhalten haben.\\ 
Der dreiphasige Vergleich mit Plecs und dem Laboraufbau hat gezeigt, dass die Amplitudenspektren Ähnlichkeit miteinander haben. Vergleicht man jedoch die Werte der beiden Spektren, ist vor allem beim sanften und harten Auf- und Absteuern eine grosse Differenz erkennbar. Der Grund dafür ist, dass sich die Steilheit des Auf- und Absteuern beim praktischen Aufbau anders verhält als bei der theoretischen Simulation.\\
Die Kombination der beiden Ansteuerungsverfahren wurde mit Plecs im ein- und dreiphasigen System aufgebaut. Mit dem Arduino und der selber geschriebenen Software, konnte sie beim Laboraufbau implementiert werden. Die Steilheit des Auf- und Absteuern, die Dauer der maximalen Spannung sowie die Anzahl der eingeschalteten Paketen des dreiphasigen Thyristorstellers können manuell im Programm eingestellt werden. Es stellte sich heraus, dass das sanfte Auf- und Absteuern der Leistung das geeignetste Verfahren ist, um die beiden Lasten (Widerstand und Asynchronmaschine) anzusteuern. Die Werte der Oberschwingungsströmen und -spannungen halten die Grenzwerte der Normen ein. Auch der untersuchte Leistungsfaktor hat bei dem Verfahren im Vergleich zu den anderen Ansteuerungsarten den besten Wert erhalten.\\
Beim Aufbau der Sparvarianten, mit Ansteuerung von zwei Thyristoren, wurde erkannt, dass die ASM die Grenzwerte der Oberschwingungsspannung nicht einhaltet. Beim Widerstand erfüllt sie grundsätzlich die Normen, jedoch werden die Phasen unterschiedlich stark belastet. Deshalb sind auch diese Verfahren nicht für die Ansteuerung geeignet.\\
Leider konnten, aus Zeitgründen, keine genaueren Erkenntnisse zu den Netzrückwirkung von sub- und zwischenharmonischen Oberwellen gefunden werden. Sie treten vor allem bei Steuerungsarten mit Schwingungspakete auf. Es wird jedoch vermutet, dass je näher die Seitenbänder an den jeweiligen Harmonischen sind, desto weniger störend sind sie für die elektrischen Geräte.\\
Mithilfe eines selbst gebauten Thyristorstellers wäre es möglich, alle drei Phasen unterschiedlich anzusteuern. Damit könnte das Verhalten der Halbwellensteuerung im Laboraufbau vollzogen und untersucht werden.\\
Am Ende des Projektes kann durchaus eine positive Bilanz gezogen werden. Auch wenn einige Verfahren, wie zum Beispiel die Auswirkung der sub- und zwischenharmonischen der verschiedenen Ansteuerungsarten, nicht herausgefunden wurde. Das Projekt hat gezeigt, dass es jedoch möglich ist, mit einer Kombination von zwei Steuerungsverfahren störende Oberschwingungen zu vermeiden. Zudem konnten neue und wichtige Erkenntnisse des Simulationsprogrammes Plecs realisiert werden.  





