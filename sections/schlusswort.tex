\section{Schlusswort}

Am Ende des Projektes kann durchaus eine positive Bilanz gezogen werden. Die beiden Verfahren, Phasenanschnitt- und Schwingungspaketsteuerung, konnten analytisch auf verlustbehaftete Oberwellen untersucht werden. Dabei erkannte man, dass je grösser der Winkel des Phasenanschnittes gewählt wird, desto gösser werden die Amplitudenwerte der harmonischen Oberschwingungen. Bei der Schwingungspaketsteuerung treten, bei kleinerem Duty Cycle, mehr sub- und zwischenharmonische auf. Mit Hilfe der Simulationen-Tools Matlab und Plecs konnten die einphasige analytische Untersuchung der beiden Steuerverfahren ausgeführt und verifiziert werden. Es zeigte sich das beide Tools die gleichen Werte der Amplitudenspektren herausgegeben haben. Der Vergleich der dreiphasigen Simulationen mit Plecs und dem Laboraufbau hat gezeigt, dass die Amplitudenspektren eine visuelle Ähnlichkeit haben. Vergleicht man jedoch die Werte der Amplituden ist vor allem bei der Schwingungspaketsteuerung eine grosse Abweichung erkennbar. Der Grund dafür sind, das nicht gleiche Verhalten des Auf- und Absteuern und die nicht vorhanden idealen Bauteile. Des weiteren konnte die Kombination der beiden Verfahren mit Plecs im ein- und dreiphasigen System aufgebaut werden. Anstelle eines UniPi-Steuergerät verwendete man einen Arduino, mit einer selbst geschriebener Software, um die Kombination der beiden Steuerverfahren beim Messaufbau zu implementieren. Mit ihr kann die Steilheit des Auf- und Absteuern, die Dauer des maximalen Wertes sowie die Anzahl der eingeschalteten Pakete des dreiphasigen Thyristorstellers eingestellt werden. Als Last wurde nicht nur eine ohmscher Verbraucher untersucht, sondern auch ein ohmsch-induktiver (Asynchronmaschine). Es stellte sich heraus, dass das sanfte Auf- und Absteuern der Leistung das geeignetste Verfahren ist, um den Widerstand und die Asynchronmaschine anzusteuern. Die Werte der Oberschwingungsströme und Oberschwingungsspannung halten die Grenzwerte der Normen ein. Leider konnte keine genau Erkenntnisse zu den sub- und zwischenharmonischen Oberwellen erhalten werden. Da jedoch die Seitenbänder    





