\section{Resultate des Messaufbaus}
In diesem Kapitel werden die Messresultate des Laboraufbaus analysiert und mit den Werten der Simulationen und den Normen verglichen. Hierbei wurden die Daten der Messungen als .csv Datei gespeichert und anschliessend mit Matlab dargestellt. Ausserdem ist es möglich die FFTs der Signale zu berechnen.\\
Es sind nur die Messungen des Widerstandes und der ASM, die in Stern geschaltet sind aufgelistet. Die Messungen in Dreieck sind hier nicht aufgeführt, sie  befinden sich als Matlabdateien digital auf dem USB-Stick. \todo{angeben ob als USB Stick/CD oder sonst abgegeben}

\subsection{Messungen Ströme}
Die Ströme der verschiedenen Ansteuerungarten und Verbrauchern sind auf zu- und unzulässige Oberschwingungen zu untersuchen. Sie müssen zwingend die Werte der Normen \ref{sec:Normen} einhalten. Ist dies der Fall werden anschliessend noch die Spannungen untersucht. Halten Ansteuerungsarten die Normen der Ströme nicht ein, werden die Spannungen nicht weiter behandelt. Bei den Schwingungspaketsteuerungen werden die Ströme nicht angeschaut, da die nicht von Interesse sind. Damit man einen Vergleich zur Simulation hat, ist diese Ansteuerung bei der Spannung aufgelistet. Die Vergleiche werden immer mit Hilfe der FFT-Funktion von Matlab durchgeführt. 

\subsubsection{Messungen Widerstand}

Die Resultate der Strommessungen sind wie folgt aufgebaut. Bei den Abbildungen \ref{fig:Mess_Widerstand_Phas_60grad_stroeme}, \ref{fig:Mess_Widerstand_Phas_90grad_stroeme}, \ref{fig:Mess_Widerstand_Sanft_stroeme} und \ref{fig:Mess_Widerstand_Sanft_langsam_stroeme} sind zuerst die Ströme, die durch den Culatti fliessen dargestellt. Der Culatti ist ein rein ohmscher Widerstand mit dem Wert von \SI{150}{\Omega} pro Strang. Der maximale Effektivstrom, der dabei durchgelassen werden darf beträgt \SI{2.4}{A}. Die zweite Grafik zeigt das FFT des gemessenen Stromes, jedoch nur von einer Phase, an. Da sich bei einer ohmschen Last, alle drei Phasen, gleich verhalten, werden die anderen zwei nicht angezeigt. Da die Norm \ref{sec:Normen} einen maximalen Wert von bis zu \SI{16}{A} Effektivwert behandelt, wurde der gemessene Wert auf die \SI{16}{A} hochgerechnet. Dies ist in der dritten Grafik der jeweiligen Abbildung ersichtlich. Anschliessend berechnete man von diesen Ströme das FFT und verglich die Werte der Amplitude tabellarisch mit den dazugehörigen Normen \ref{sec:Normen}.

\subsubsection*{Phasenanschnitt 60\textdegree}

In der Abbildung \ref{fig:Mess_Widerstand_Phas_60grad_stroeme} ist das Stromsignal mit einem Phasenanschnittswinkel von 60\textdegree ersichtlich. 

\begin{figure}[ht!]
	\centering
	\includegraphics[width=\textwidth]{Messung_Widerstand_Phas_60grad_stroeme.png}	
	\caption{Messung mit Phasenanschnitt 60\textdegree}\label{fig:Mess_Widerstand_Phas_60grad_stroeme}
\end{figure}


\begin{table}[ht!]
	\centering
	\begin{tabular}{|l|l|l|}
		\hline
		Oberschwingungsordnung & Amplitude [A] 	& Verhältnis zur Grundschwingung	\\ \hline
		1                      & 21.1996   		& 100\%								\\ \hline
		5                      & 3.7851    		& 17.86\%							\\ \hline
		7                      & 2.6127    		& 12.32\%							\\ \hline
		11                     & 1.2267    		& 5.79\%							\\ \hline
	\end{tabular}
	\caption{Amplitudenwerte bei den harmonischen Oberschwingungen bei Phasenanschnitt 60\textdegree}\label{tab:Phas_60_Stroeme}
\end{table}
%Die Höhe der Amplituden der Oberschwingungen der Tabelle \ref{tab:Phas_60_Stroeme} können mit den Normen in der Tabelle \ref{tab:Grenzwerte_Normen} verglichen werden. Dabei wird festgestellt, dass die Werte der Messung höher sind als die Normen zulassen.

Wenn die Werte der Messungen in der Tabelle \ref{tab:Phas_60_Stroeme} mit der Werte der Normen in der Tabelle \ref{tab:Grenzwerte_Normen} verglichen werden, ist ersichtlich, dass die Amplitudenwerte bei der 5. bis 11. Oberschwingungsordnung zu hoch sind. Es zeigt, dass sich der Phasenanschnitt mit 60\textdegree \hspace{0.02cm} nicht eignet um den Widerstand anzusteuern, da dies nicht den Normen entspricht. Die Spannungen werden deshalb nicht mehr untersucht. 


\subsubsection*{Phasenanschnitt 90\textdegree}

Die nächst Abbildung \ref{fig:Mess_Widerstand_Phas_90grad_stroeme} zeigt die Ansteuerung mit einem Phasenanschnitt von 90\textdegree \hspace{0.02cm}. 

\begin{figure}[ht!]
	\centering
	\includegraphics[width=\textwidth]{Messung_Widerstand_Phas_90grad_stroeme.png}	
	\caption{Phasenanschnitt 90\textdegree}\label{fig:Mess_Widerstand_Phas_90grad_stroeme}
\end{figure}

\begin{table}[ht!]
	\centering
	\begin{tabular}{|l|l|l|}
		\hline
		Oberschwingungsordnung 	& Amplitude [A] & Verhältnis zur Grundschwingung	\\ \hline
		1       				& 14.647   		& 100\%								\\ \hline
		5      					& 6.3481    	& 43.34\%							\\ \hline
		7      					& 3.2571    	& 22.24\%							\\ \hline
		11      				& 2.273    		& 15.52\%							\\ \hline
	\end{tabular}
	\caption{Amplitudenwerte bei den harmonischen Oberschwingungen bei Phasenanschnitt 90\textdegree}\label{tab:Phas_90_Stroeme}
\end{table}

Die Amplitudenwerte der 5. bis 11. Oberschwingungsanordnung, erkennbar in der Tabelle \ref{tab:Phas_90_Stroeme}, sind auch bei dieser Ansteuerung, im Vergleich zu den Normen \ref{tab:Grenzwerte_Normen}, zu hoch. Somit eignet sich auch der Phasenanschnitt mit 90\textdegree \hspace{0.02cm} nicht, den Widerstand anzusteuern. Auf die Spannung wird auch hier nicht mehr eingegangen. 

%Wenn die Werte mit der Werte der Tabelle \ref{tab:Grenzwerte_Normen} verglichen werden, ist ersichtlich, dass die Amplituden bei der Oberschwingungsordung 5 bis 19 zu hoch sind. So kann gesagt werden, dass sich der Phasenanschnitt mit 90\textdegree \hspace{0.02cm} nicht eignet um Widerstände anzusteuern, da diese nicht den Normen entsprechen. Auf der Abbildung \ref{fig:Mess_Widerstand_Phas_90grad_stroeme} ist ersichtlich, dass bei den Oberschwingungsordnungen 3, 9 und 15 die Amplitude 0 ist, deswegen wurde diese nicht in der Tabelle \ref{tab:Phas_60_Stroeme} aufgeführt.


\newpage
\subsubsection*{Hartes Auf- und Absteuern}

In der Abbildung \ref{fig:Mess_Widerstand_Sanft_stroeme} ist das Harte Auf- und Absteuern des Widerstands erkennbar. 

\begin{figure}[ht!]
	\centering
	\includegraphics[width=\textwidth]{Messung_Widerstand_Sanft_stroeme.png}	
	\caption{Messung mit Auf- und Absteuern}\label{fig:Mess_Widerstand_Sanft_stroeme}
\end{figure}


\begin{table}[ht!]
	\centering
	\begin{tabular}{|l|l|l|}
		\hline
		Frequenz {[}Hz{]} & Amplitude {[}A{]} & Verhältnis zur Grundschwingung	\\ \hline
		49.3              & 1.5146            & 20.51\%							\\ \hline
		49.6              & 4.4853            & 60.73\%							\\ \hline
		49.7              & 2.618             & 35.45\%							\\ \hline
		50                & 7.3857            & 100\%							\\ \hline
		50.05             & 3.73              & 50.5\%							\\ \hline
		50.35             & 4.662             & 63.12\%							\\ \hline
		50.7              & 1.5504            & 21\%							\\ \hline
		248.65            & 0.6226            & 8.43\%							\\ \hline
		250               & 0.0883            & 1.2\%							\\ \hline
		251.45            & 0.6               & 8.12\%							\\ \hline
	\end{tabular}
	\caption{Amplitudenwerte bei verschiedenen Frequenzen beim hartem Auf- und Absteuern}\label{tab:Sanft_stroeme}
\end{table}
Da beim harten Auf- und Absteuern bereits die 5. Oberschwingungsanordnung, einen Amplitudenwert von unter \SI{0.1}{A} hat, wurde darauf verzichtet, weitere Werte der harmonischen Schwingungen tabellarisch aufzulisten. Jedoch sind bei dieser Messung die Sub- und Zwischenharmonische sehr interessant. Diese sind in der Tabelle \ref{tab:Sanft_stroeme} aufgeführt. Es ist ersichtlich, dass die Werte der Sub- und Zwischenharmonischen um \SI{50}{Hz}, mehr als die Hälfte der Grundschwingung entsprechen. Diese sogenannten Trägerbänder sind auch bei den weiteren harmonischen Oberwellen erkennbar. Sie sind jedoch kleiner als die, um die \SI{50}{Hz}. Vergleicht man die Werte der Amplitude der harmonischen Schwingungen mit den Normen, sind sie im Bereich der erlaubten Grenzwerte. Deshalb wird noch eine Untersuchung der Spannung vorgenommen.


\newpage
\subsubsection*{Sanftes Auf- und Absteuern}\label{sec:Sanft_Widerstand_stroeme}
In Abbildung \ref{fig:Mess_Widerstand_Sanft_langsam_stroeme} erkennt man ein sanftes Auf- und Absteuern der Ströme.

\begin{figure}[ht!]
	\centering
	\includegraphics[width=\textwidth]{Messung_Widerstand_Sanft_langsam_stroeme.png}	
	\caption{Messung mit sanftem Auf- und Absteuern}\label{fig:Mess_Widerstand_Sanft_langsam_stroeme}
\end{figure}

\begin{table}[ht!]
	\centering
	\begin{tabular}{|l|l|l|}
		\hline
		Frequenz {[}Hz{]} & Amplitude {[}A{]} & Verhältnis zur Grundschwingung	\\ \hline
		49.8              & 1.148             & 9.8\%							\\ \hline
		49.85             & 1.786             & 15.25\%							\\ \hline
		49.9              & 1.519             & 12.97\%							\\ \hline
		49.95             & 6.703             & 57.22\%							\\ \hline
		50                & 11.715            & 100\%							\\ \hline
		50.05             & 7.136             & 60.91\%							\\ \hline
		50.1              & 1.473             & 12.57\%							\\ \hline
		50.15             & 1.923             & 16.41\%							\\ \hline
		249.65            & 0.563             & 4.81\%							\\ \hline
		250               & 0.186             & 1.59\%							\\ \hline
		250.35            & 0.559             & 4.77\%							\\ \hline
	\end{tabular}
	\caption{Amplitudenwerte bei den harmonischen Oberschwingungen bei sanftem Auf- und Absteuern}\label{tab:Sanft_langsam_stroeme}
\end{table}
Es ist zu erkennen, dass auch hier fast keine harmonische Oberwellen vorhanden sind. Die Amplitude der Grundschwingung ist jedoch höher als beim harten Auf- und Absteuern. Dies hat den Grund, dass länger auf der vollen Leistung gefahren wurde. Beim sanften Auf- und Absteuern ist das Hoch- und Runterfahren langsamer, deshalb ist das Trägerband näher bei \SI{50}{Hz} als beim harten Auf- auf und Absteuern \ref{fig:Mess_Widerstand_Sanft_stroeme}. Im Vergleich zum harten Auf- und Absteuern, ist die 5. Harmonische 0.39\% grösser, welches einem sehr kleinen Unterschied entspricht. Die Peaks des Trägerbandes bei \SI{250}{Hz} sind bei dem sanften Auf- und Absteuern kleiner als die beim harten Auf- und Absteuern.



\newpage
\subsubsection{Messungen ASM}
Die Strommessungen der ASM wurde mit den bekannten Phasenanschnittswinkeln und mit dem sanften Auf- und Absteuern vorgenommen. Dies hat den Grund, da die Anwendung für das harte Auf- und Absteuern oder der Schwingungspaketsteuerung nicht vorkommt. Ausserdem wird in der Praxis meistens der Phasenanschnitt verwendet. Wie auch schon beim Widerstand muss der Strom bei der ASM auf die \SI{16}{A} hochgerechnet werden, um diese mit den Norm tabellarisch vergleichen zu können. Die Messungen des Asynchronmotors sind ähnlich aufgebaut wie die des Widerstands. Einzig ein Spannungssignal des Reglers wurde hinzugefügt. Es ist erkennbar, dass das jeweilige Hochfahren des Stromes einen Einfluss auf das Spannungssignal des Drehgeber hat. Je harter man Hochfährt desto steiler wird die Kurve des Spannungssignal des Drehgeber. 


\subsubsection*{Phasenaschnitt 60\textdegree}
In Abbildung \ref{fig:Mess_Phas_60grad_stroeme} erkennt man ein einen Phasenanschnitt von 60\textdegree der Ströme.

\begin{figure}[ht!]
	\centering
	\includegraphics[width=\textwidth]{Messung_ASM_Phas_60grad_stroeme.png}	
	\caption{Messung mit Phasenanschnitt 60\textdegree}\label{fig:Mess_Phas_60grad_stroeme}
\end{figure}

\begin{table}[ht!]
	\centering
	\begin{tabular}{|l|l|l|}
		\hline
		Frequenz {[}Hz{]} & Amplitude {[}A{]} & Verhältnis zur Grundschwingung	\\ \hline
		49.7              & 0.76              & 35.53\%							\\ \hline
		49.9              & 1.317             & 61.57\%							\\ \hline
		50                & 2.139             & 100\%							\\ \hline
		50.1              & 1.53              & 71.53\%							\\ \hline
		50.3              & 0.675             & 31.56\%							\\ \hline
		250               & 0.07              & 3.27\%							\\ \hline
	\end{tabular}
	\caption{Amplitudenwerte bei den harmonischen Oberschwingungen bei Phasenanschnitt 60\textdegree}\label{tab:Phas_60_ASM_stroeme}
\end{table}
<<<<<<< HEAD
Entgegen den Vorkenntnissen bei Phasenanschnitt mit einem Winkel von 60\textdegree, gibt es bei der Ansteuerung der ASM praktisch keine harmonischen Oberwellen sondern fast nur sub- und zwischenharmonische. Dies hat den Grund, dass ein reiner Sinus mit einer Frequenz von \SI{50}{Hz} bei der maximalen Drehzahl entsteht. Die 5. harmonische Schwingung hat eine Amplitude von \SI{0.07}{A} und hält somit die Normen ein. Weitere Harmonische sind keine zu erkennen. Die Peaks des Trägerbandes um die Grundschwingung sind im Verhältnis mit bis zu 70\% sehr hoch.
=======
ntgegen den Vorkenntnissen bei Phasenanschnitt mit einem Winkel von 60\textdegree, gibt es bei der Ansteuerung der ASM praktisch keine harmonischen Oberwellen sondern fast nur sub- und zwischenharmonische. Dies hat den Grund, dass ein reiner Sinus von einer Frequenz von \SI{50}{Hz} bei der maximalen Drehzahl entsteht. Die fünft harmonische Schwingung hat eine Amplitude von \SI{0.07}{A} und hält somit die Normen ein. Weiter Harmonische sind keine zu erkennen. Die Peaks des Trägerbandes um die Grundschwingung sind im Verhältnis mit bis zu 70\% sehr hoch.
>>>>>>> master

\newpage
\subsubsection*{Phasenaschnitt 90\textdegree}

In Abbildung \ref{fig:Mess_Phas_90grad_stroeme} erkennt man ein einen Phasenanschnitt von 90\textdegree der Ströme.

\begin{figure}[ht!]
	\centering
	\includegraphics[width=\textwidth]{Messung_ASM_Phas_90grad_stroeme}	
	\caption{Messung mit Phasenanschnitt 90\textdegree}\label{fig:Mess_Phas_90grad_stroeme}
\end{figure}
<<<<<<< HEAD
  

=======
 
>>>>>>> master
\begin{table}[ht!]
	\centering
	\begin{tabular}{|l|l|l|}
		\hline
		Frequenz {[}Hz{]} & Amplitude {[}A{]} & Verhältnis zur Grundschwingung	\\ \hline
		49.7              & 1.399             & 42.6\%							\\ \hline
		49.9              & 2.023             & 61.61\%							\\ \hline
		50                & 3.2839            & 100\%							\\ \hline
		50.1              & 2.567             & 78.17\%							\\ \hline
		50.3              & 1.468             & 44.71\%							\\ \hline
		250               & 0.673             & 20.5\%							\\ \hline
		250.1             & 0.959             & 29.2\%							\\ \hline
		250.2             & 0.687             & 20.92\%							\\ \hline
	\end{tabular}
	\caption{Amplitudenwerte bei den harmonischen Oberschwingungen bei Phasenanschnitt 90\textdegree}\label{tab:Phas_90_ASM_stroeme}
\end{table}

Anders als bei der Ansteuerung mit dem Phasennaschnittswinkel von 60\textdegree, treten mit 90\textdegree \hspace{0.02cm}, bei der 5. Harmonischen, grössere sub-, zwischenharmonische und harmonische Oberwellen auf. Auf der Abbildung \ref{fig:Mess_Phas_90grad_stroeme} ist beim Stromsignal ersichtlich, dass die Maschine schwingt. Beim Testen der ASM stellte man zudem fest, dass die ASM nicht mit konstant gleicher Drehzahl dreht.
Im Vergleich zum Phasenanschnitt mit 60\textdegree, sind die Peaks des Trägerbandes bei der Grundschwingung noch höher. Auch die 5. Harmonische entspricht im Verhältnis zur Grundschwingung 20.5\%. Der Wert der Amplitude ist jedoch immer noch kleiner, als der erlaubte Grenzwert der Norm. 


\newpage
\subsubsection*{Sanftes Auf- und Absteuern}

In Abbildung \ref{fig:Mess_Sanft_langsam_stroeme} erkennt man ein einen sanftes Auf- und Absteuern der Ströme.

\begin{figure}[ht!]
	\centering
	\includegraphics[width=\textwidth]{Messung_ASM_Sanft_langsam_stroeme.png}	
	\caption{Messung mit sanftem Auf- und Absteuern}\label{fig:Mess_Sanft_langsam_stroeme}
\end{figure}


\begin{table}[ht!]
	\centering
	\begin{tabular}{|l|l|l|}
		\hline
		Frequenz {[}Hz{]} & Amplitude {[}A{]} & Verhältnis zur Grundschwingung	\\ \hline
		49.8              & 0.823             & 20.51\%							\\ \hline
		49.9              & 1.064             & 26.51\%							\\ \hline
		49.95             & 1.716             & 42.76\%							\\ \hline
		50                & 4.013             & 100\%							\\ \hline
		50.05             & 1.613             & 40.19\%							\\ \hline
		50.1              & 1.141             & 28.43\%							\\ \hline
		50.2              & 0.878             & 21.88\%							\\ \hline
		250               & 0.28              & 6.98\%							\\ \hline
	\end{tabular}
	\caption{Amplitudenwerte bei den harmonischen Oberschwingungen bei sanftem Auf- und Absteuern}\label{tab:Sanft_langsam_ASM_stroeme}
\end{table}
<<<<<<< HEAD

Wie schon bei der Widerstandsmessung mit dem sanftem Auf- und Absteuern \ref{sec:Sanft_Widerstand_stroeme}, sind bei dieser Messung, mit der ASM, die harmonische Oberschwingungen sehr klein. Sie halten somit die Norm ein. In der Tabelle \ref{tab:Sanft_langsam_ASM_stroeme} sind bei verschiedenen Frequenzen, die Werte der Amplituden des FFTs aufgelistet.
=======
Wie schon bei der Widerstandsmessung mit dem sanftem Auf- und Absteuern\ref{sec:Sanft_Widerstand_stroeme}, sind bei dieser Messung, mit der ASM, die harmonische Oberschwingungen sehr klein. Sie halten somit die Norm ein. In der Tabelle \ref{tab:Sanft_langsam_ASM_stroeme} sind bei verschiedenen Frequenzen die Werte der Amplituden des FFTs aufgelistet.
>>>>>>> master
Die Peaks des Trägerbandes um \SI{50}{Hz} haben ein Verhältnis von bis zu 40\% zur Grundfrequenz.


\newpage
\subsection{Messungen Spannungen}
Damit man die Funktionen des Laboraufbaues mit den Simulationen vergleichen kann, wurden die Spannungen über dem Widerstand und dem Asynchronmotor mit den verschiedenen Ansteuerungsarten gemessen. Dafür wurden die Spannungssignale und das FFT als Grafik und die interessanten Werte der sub-, zwischnenharmonischen und harmonischen Schwingungen tabellarisch aufgelistet. Damit auch die Spannungen mit den Normen verglichen werden können, wurde das Verhältnis der verschiedenen Oberschwingungen zur Grundschwingung in Prozent in die Tabellen eingefügt. Da die Phasenantschnittsteureungen die Normen der harmonischen Oberwellen des Stromes nicht einhalten, werden die Spannungssignale bei diesen Verfahren nicht mehr aufgezeigt. Sie sind jedoch im Anhang \ref{sec:Mess_Spannung_Widerstand} ersichtlich. Die Abbildungen in diesem Kapitel sind so aufgebaut, dass zuerst das Spannungssignal über dem Widerstand und anschliessend das daraus berechnete FFT dargestellt wurde.

\subsubsection{Messungen Widerstand}
Für die Spannung über dem Widerstand wurden die Schwingungspaketsteuerung mit einem Duty Cycle von 0.5 und 0.8 gemessen. Danach wurde das harte und sanfte Auf- und Absteuern des Widerstand untersucht. Folgende Resultate haben sich ergeben:


\subsubsection*{Schwingungspaket mit Duty Cycle von 0.5}

In Abbildung \ref{fig:Mess_Schwing_50} ist die Schwingungspaketsteuerung mit einem Duty Cycle von 0.5 dargestellt.


\begin{figure}[ht!]
	\centering
	\includegraphics[width=\textwidth]{Messung_Widerstand_Schwing_0_5.png}	
	\caption{Messung mit Schwingungspaket 50\%}\label{fig:Mess_Schwing_50}
\end{figure}


\newpage
\begin{table}[ht!]
	\centering
	\begin{tabular}{|l|l|l|}
		\hline
		Frequenz {[}Hz{]} & Amplitude {[}V{]} & Verhältnis zur Grundschwingung \\ \hline
		46                & 14.7351           & 9.83\%                         \\ \hline
		47                & 28                & 18.68\%                        \\ \hline
		48                & 26.376            & 17.59\%                        \\ \hline
		49                & 95.6              & 63.77\%                        \\ \hline
		50                & 149.92            & 100\%                          \\ \hline
		51                & 99.8              & 66.57\%                        \\ \hline
		52                & 25.134            & 16.76\%                        \\ \hline
		53                & 20.6              & 13.74\%                        \\ \hline
	\end{tabular}
\caption{Amplitudenwerte bei der Frequenzen bei Schwingungspaket 50\%}\label{tab:Mess_Spannung_Schwing_50}
\end{table}
Wie bei den Simulationen im Kapitel \ref{sec:Schwingungspaketsteuerung_Simulation} gezeigt, treten bei der Schwingungspaketsteuerung keine harmonische Oberwelle auf. Bei dem FFT erkennt man, dass das Trägerband bei der Grundschwingung relativ breit ist. In der Tabelle \ref{tab:Mess_Spannung_Schwing_50} sind die Werte der Amplituden der sub- und zwischenharmonischen aufgelistet. Dabei wurden die Werte aus dem FFT ausgelesen, welche sich in der nähe der Grundschwingung befinden und eine hohe Amplitude besitzen. 
Es ist in der Tabelle ersichtlich, dass die beiden Frequenzen neben der Grundschwingung mehr als 60\% so hoch sind wie die Amplitude bei \SI{50}{Hz}. Selbst \SI{4}{Hz} neben der Grundschwingung beträgt das Verhältnis immer noch fast 10\%.\\
Die Werte der Tabelle \ref{tab:Mess_Spannung_Schwing_50} wurden mit denen des FFTs der Plecs-Simulation verglichen. Die Grafik und die Werten der Messung und der Simulation befinden sich im Anhang \ref{sec:Vergleich_Mess_Sim_Schwing_50} . Dabei wurde festgestellt, dass die Kurvenformen der FFT zwar ähnlich aussehen. Bei näherem betrachten gibt jedoch grosse Unterschiede zwischen den Peak-Werten. In der Simulation schnellen, nach dem Einschalten eines Paketes, die Spannungen sogleich an den Spitzenwert, wobei dies beim Laboraufbau nicht der Fall ist. Deshalb ist bei der Simulation, bei gleicher Zeitdauer, die Spannung länger auf dem Spitzenwert als beim Laboraufbau. Des weiteren sind alle Bauteile in der Simulation ideal wobei dies in der Praxis nicht der Fall ist. Deshalb ist auch eine minimale Abweichung des Spitzenwertes ersichtlich. Die Standartabweichung aller aufgeführten Frequenzen beträgt: 7.912. Dieser Wert wurde mit Excel berechnet wobei die Frequenzen als Stichprobe angenommen wurden. 


<<<<<<< HEAD
Die Werte der Tabelle \ref{tab:Mess_Spannung_Schwing_50} wurden mit denen des FFTs der Plecs-Simulation verglichen. Die Grafik und die Werten der Messung und der Simulation befinden sich im Kapitel \ref{sec:Vergleich_Mess_Sim_Schwing_50} im Anhang. Dabei wurde festgestellt, dass die Kurvenformen zwar ähnlich aussehen jedoch bei näherem betrachten gibt es grosse Unterschiede. In der Simulation schnellen nach dem Einschalten eines Paketes die Spannungen sogleich an ihren Spitzenwert wobei dies beim Laboraufbau nicht der Fall ist. 
Zwar wurde die Zeitdauer beim Laboraufbau gleich gewählt wie bei der Simulation aber war die Spannung nicht gleich lang auf dem Spitzenwert. Zusätzlich ist der Spitzenwert nicht bei beiden gleich hoch. Des weiteren sind alle Bauteile in der Simulation ideal wobei dies in der Praxis nicht der Fall ist. Die Standartabweichung aller aufgeführten Frequenzen beträgt: 7.912. Dieser Wert wurde mit Excel berechnet wobei die Frequenzen als Stichprobe angenommen wurden.

=======
>>>>>>> master
\newpage
\subsubsection*{Schwingungspaket mit Duty Cycle von 0.8}
In der Abbildung \ref{fig:Mess_Schwing_80} ist die Schwingungspaketsteuerung mit einem Duty Cycle von 0.8 ersichtlich.
\begin{figure}[ht!]
	\centering
	\includegraphics[width=\textwidth]{Messung_Widerstand_Schwing_0_8.png}	
	\caption{Messung mit Schwingungspaket 80\%}\label{fig:Mess_Schwing_80}
\end{figure}


\begin{table}[ht!]
	\centering
	\begin{tabular}{|l|l|l|}
		\hline
		Frequenz {[}Hz{]} & Amplitude {[}V{]} & Verhältnis zur Grundschwingung \\ \hline
		46                & 20.173            & 7.58\%                         \\ \hline
		47                & 28.26             & 10.62\%                        \\ \hline
		48                & 40.576            & 15.26\%                        \\ \hline
		49                & 62.694            & 23.57\%                        \\ \hline
		50                & 265.98            & 100\%                          \\ \hline
		51                & 65.7              & 24.7\%                         \\ \hline
		52                & 43.812            & 16.47\%                        \\ \hline
		53                & 21.939            & 8.25\%                         \\ \hline
	\end{tabular}
\caption{Amplitudenwerte bei der Frequenzen bei Schwingungspaket 80\%}\label{tab:Mess_Spannung_Schwing_80}
\end{table}

<<<<<<< HEAD
<<<<<<< HEAD
Anders als beim Schwingungspaket mit einem Duty Cycle von 0.5, ist bei 0.8 der Peak bei der Grundfrequenz von von \SI{50}{Hz} deutlich höher. Dies macht Sinn, da die Spannung einen länger Zeit auf dem Maximum ist.
In der Tabelle \ref{tab:Mess_Spannung_Schwing_80} befinden sich die Werte der Amplituden und deren Verhältnis zur Grundschwingung.\\
Wie bei der Schwingungspaketsteuerung mit Duty Cycle von 0.5 wurden auch die Resultate der Messung mit 0.8, mit der Simulation verglichen. Die Tabelle mit den Werten und der grafische Vergleich befindet sich im Anhang \ref{sec:Vergleich_Mess_Sim_Schwing_80}. Auch hier gab es Abweichungen bei den verschiedenen Frequenzen zwischen der Simulation und der Messung. Diese waren jedoch bedeutend kleiner als bei der anderen Schwingungspaketsteuerung. Es resultierte eine Standartabweichung von 2.481. Als Grund für die niedrigere Abweichung ist, dass das Schwingungspakete eine längere Zeit auf dem Maximum bleibt und so eine grössere Ähnlichkeit mit der Simulation hat.
=======
Wenn die Amplituden der Frequenzen \SI{49}{Hz} und \SI{51}{Hz} im Verhältnis zur Grundschwingung viel kleiner sind als bei der Schwingungspaketsteuerung mit 50\%. Die anderen aufgelisteten Frequenzen sind im Verhältnis zur Grundschwingung etwa gleich gross.\\


Wie bei der Schwingungspaketsteuerung mit 50\% wurden bei 80\% die Resultate der Messung und der Simulation verglichen. Die Tabelle mit den Werten und der grafische Vergleich befindet sich im Anhang in Kapitel \ref{sec:Vergleich_Mess_Sim_Schwing_80}. Auch hier gab es Abweichungen bei den verschiedenen Frequenzen zwischen der Simulation und der Messung. Diese waren aber bedeutend kleiner als bei dem Schwingungspaket mit 50\%, welches in einer kleineren Standartabweichung von 2.481 resultierte. Als Grund für die Abweichung gelten die gleichen Gründe, wie bei der Schwingungspaketsteuerung mit 50\%.

>>>>>>> master
=======
Anders als beim Schwingungspaket mit einem Duty Cycle von 0.5, ist bei 0.8 der Peak bei der Grundfrequenz von von \SI{50}{Hz} deutlich höher. Dies macht Sinn, da die Spannung einen länger Zeit auf dem Maximum ist.
In der Tabelle \ref{tab:Mess_Spannung_Schwing_80} befinden sich die Werte der Amplituden und deren Verhältnis zur Grundschwingung.\\
Wie bei der Schwingungspaketsteuerung mit Duty Cycle von 0.5 wurden auch die Resultate der Messung mit 0.8, mit der Simulation verglichen. Die Tabelle mit den Werten und der grafische Vergleich befindet sich im Anhang \ref{sec:Vergleich_Mess_Sim_Schwing_80}. Auch hier gab es Abweichungen bei den verschiedenen Frequenzen zwischen der Simulation und der Messung. Diese waren jedoch bedeutend kleiner als bei der anderen Schwingungspaketsteuerung. Es resultierte eine Standartabweichung von 2.481. Als Grund für die niedrigere Abweichung ist, dass das Schwingungspakete eine längere Zeit auf dem Maximum bleibt und so eine grössere Ähnlichkeit mit der Simulation hat.

>>>>>>> master


\newpage
\subsubsection*{Hartes Auf- und Absteuern}
Die Abbildung \ref{fig:Mess_Sanft} zeigt, ein hartes Auf- und Absteuern der Spannung.

\begin{figure}[ht!]
	\centering
	\includegraphics[width=\textwidth]{Mess_Widerstand_AufAb.png}	
	\caption{Messung mit Auf- und Absteuern}\label{fig:Mess_Sanft}
\end{figure}


\begin{table}[ht!]
	\centering
	\begin{tabular}{|l|l|l|}
		\hline
		Frequenz {[}Hz{]} & Amplitude {[}V{]} & Verhältnis zur Grundschwingung \\ \hline
		49.65             & 67.126            & 60.11\%                        \\ \hline
		49.7              & 40.9583           & 36.68\%                        \\ \hline
		50                & 111.6763          & 100\%                          \\ \hline
		50.05             & 58.2021           & 52.12\%                        \\ \hline
		50.35             & 70.0651           & 62.74\%                        \\ \hline
		249               & 9.0297            & 8.09\%                         \\ \hline
		250               & 1.0487            & 0.94\%                         \\ \hline
		251 		      & 1.8206            & 1.63\%                         \\ \hline
	\end{tabular}
\caption{Amplitudenwerte bei der Frequenzen bei hartes Auf- und Absteuern}\label{tab:Mess_Spannung_AufAb_hart}
\end{table}

<<<<<<< HEAD
<<<<<<< HEAD
Für die Tabelle \ref{tab:Mess_Spannung_AufAb_hart} wurden die höchsten Amplitudenwerte bei der Grundschwingung von \SI{50}{Hz} und bei der fünften Harmonische \SI{250}{Hz} aufgelistet.
Wie auch bei den Schwingungspaketsteuerungen, tritt bei dem harten Auf- und Absteuern die harmonischen Oberschwingung praktisch nicht mehr auf. Die fünfte Harmonische besitzt nur ein Verhältnis von unter einem Prozent der Grundschwingung. Jedoch sind auch hier sub- und zwischenharmonische Oberwellen ersichtlich. Diese betragen bei den Frequenzen \SI{49.65}{Hz} und \SI{50.35}{Hz} über 60\%. Der Vergleich mit der Simulation ist im Anhang ersichtlich.
=======
Wie auch bei den Schwingungspaketsteuerungen, tritt bei dem harten Auf- und Absteuern die harmonischen Oberschwingung praktisch nicht mehr auf. Die fünfte Harmonische besitzt nur ein Verhältnis von unter einem Prozent der Grundschwingung. Jedoch treten auch hier wieder die Sub- und Zwischenharmonische auf. Diese betragen bei den Frequenzen \SI{49.65}{Hz} und \SI{50.35}{Hz} über 60\%. \\
Für den Vergleich mit dem harten Auf- und Absteuern gibt es einen Unterschied der Signaldauern. Während bei der Simulation das Auffahren \SI{0.2}{s} dauert, benötigt der Laboraufbau mit fast \SI{0.8}{s} deutlich länger. Ein weiterer Unterschied waren bei der Messung die Zeitdauer zwischen den verschiedenen Auf- und Absteuerpakete. Da der Thyristorsteller nicht sehr schnell reagiert, dauerte es fast \SI{1}{s} bis das nächste Paket anfangen konnte. Diese Gründe erklären den grossen Unterschied zwischem dem FFT der Simulation und der Messung. Berechnet wurde eine Standartabweichung von 19.812 wobei dieser Wert 2.5 mal gross ist, wie die Standartabweichung des Schwingungspaketes mit 50\%.


>>>>>>> master
=======


Für den Vergleich mit dem harten Auf- und Absteuern gibt es einen Unterschied der Signaldauern. Während bei der Simulation das Auffahren \SI{0.2}{s} dauert, benötigt der Laboraufbau mit fast \SI{0.8}{s} deutlich länger. Ein weiterer Unterschied waren bei der Messung die Zeitdauer zwischen den verschiedenen Auf- und Absteuerpakete. Da der Thyristorsteller nicht sehr schnell reagiert, dauerte es fast \SI{1}{s} bis das nächste Paket anfangen konnte. Diese Gründe erklären den grossen Unterschied zwischem dem FFT der Simulation und der Messung. Berechnet wurde eine Standartabweichung von 19.812 wobei dieser Wert 2.5 mal gross ist, wie die Standartabweichung des Schwingungspaketes mit 50\%.
\todo{Satz behalten?}

Für die Tabelle \ref{tab:Mess_Spannung_AufAb_hart} wurden die höchsten Amplitudenwerte bei der Grundschwingung von \SI{50}{Hz} und bei der fünften Harmonische \SI{250}{Hz} aufgelistet.
Wie auch bei den Schwingungspaketsteuerungen, tritt bei dem harten Auf- und Absteuern die harmonischen Oberschwingung praktisch nicht mehr auf. Die fünfte Harmonische besitzt nur ein Verhältnis von unter einem Prozent der Grundschwingung. Jedoch sind auch hier sub- und zwischenharmonische Oberwellen ersichtlich. Diese betragen bei den Frequenzen \SI{49.65}{Hz} und \SI{50.35}{Hz} über 60\%. Der Vergleich mit der Simulation ist im Anhang ersichtlich.
>>>>>>> master

\newpage
\subsubsection*{Sanftes Auf- und Absteuern}
Die Abbildung \ref{fig:Mess_Sanft_langsam} zeigt, ein sanftes Auf- und Absteuern der Spannung.


\begin{figure}[ht!]
	\centering
	\includegraphics[width=\textwidth]{Messung_Widerstand_Sanft_langsam.png}	
	\caption{Messung mit sanftem Auf- und Absteuern}\label{fig:Mess_Sanft_langsam}
\end{figure}


\begin{table}[ht!]
	\centering
	\begin{tabular}{|l|l|l|}
		\hline
		Frequenz {[}Hz{]} & Amplitude {[}V{]} & Verhältnis zur Grundschwingung \\ \hline
		49.8              & 18.522            & 10.75\%                        \\ \hline
		49.85             & 26.576            & 15.43\%                        \\ \hline
		49.9              & 29.507            & 17.131\%                       \\ \hline
		49.95             & 91.266            & 52.99\%                        \\ \hline
		50                & 172.241           & 100\%                          \\ \hline
		50.05             & 116.719           & 67.76\%                        \\ \hline
		50.1              & 28.629            & 16.62\%                        \\ \hline
		50.15             & 30.076            & 17.46\%                        \\ \hline
		50.2              & 18.72             & 10.87\%                        \\ \hline
		249.6             & 8.183             & 4.75\%                         \\ \hline
		250               & 1.158             & 0.67\%                         \\ \hline
		250.4             & 7.466             & 4.33\%                         \\ \hline
	\end{tabular}
\caption{Amplitudenwerte bei der Frequenzen bei Sanftes Auf- und Absteuern}\label{tab:Mess_Spannung_AufAb_sanft}
\end{table}

<<<<<<< HEAD
<<<<<<< HEAD
Im visuellen Vergleich mit dem hartem Auf- und Abfahren zeigt das FFT, dass die Frequenzbänder dünner geworden sind und der Peak bei \SI{50}{Hz} grösser. In der Tabelle \ref{tab:Mess_Spannung_AufAb_sanft} sind die höchsten Amplitudenwerte der Frequenzen, die sich in der nähe der Grundschwingung von \SI{50}{Hz} und der fünften Harmonischen \SI{250}{Hz} aufhalten, aufgelistet. \todo{Einfügen Vergleich Simulation}

Der Vergleich mit der Tabelle \ref{tab:Mess_Spannung_AufAb_hart} zeigt, dass die höchsten Peaks nicht mehr bei den sub- und zwischenharmonischen bei den Frequenzen von \SI{49.65}{Hz} und \SI{50.35}{Hz} liegen, sonder bei \SI{49.95}{Hz} und \SI{50.05}{Hz}. Das Verhältnis bei \SI{250}{Hz} zur Grundschwingung ist beim sanftem Auf- und Abfahren kleiner geworden als beim harten Auf- und Abfahren. Wie auch bei \SI{50}{Hz} sind die höchsten Amplituden näher an der harmonische Schwingung. Der Vergleich zur Simulation erkennt man im Anhang.
=======
Der Vergleich mit der Tabelle \ref{tab:Mess_Spannung_AufAb_hart} zeigt auf, dass die höchsten Peaks bei den Sub- und Zwischenharmonischen nicht mehr bei den Frequenzen \SI{49.65}{Hz} und \SI{50.35}{Hz} liegen, sonder näher bei der Grundschwingung liegen, bei \SI{49.95}{Hz} und \SI{50.05}{Hz}. Das Verhältnis bei \SI{250}{Hz} zur Grundschwingung ist beim sanftem Auf- und Abfahren kleiner geworden als beim harten Auf- und Abfahren. Aber wie auch bei \SI{50}{Hz} sind die höchsten Amplituden näher an der harmonische Schwingung. \\
Der Vergleich mit der Simulation und der Messung für den sanften Auf- und Abfahren, welches sich im Kapitel \ref{sec:Vergleich_Mess_Sim_sanft_AufAb} im Anhang befindet, zeigt grosse Unterschiede bei den Amplituden auf. Ein grosser Grund dafür ist, dass bei der Simulation das Hoch- und Runterfahren eine Zeitdauer von \SI{0.3}{s} haben, wobei für die Messung diese Zeitdauer \SI{3}{s} beträgt. Zusätzlich ist bei der Simulation während \SI{6}{s} die Spannung auf dem Spitzenwert, beim Laboraufbau sind es eher \SI{7}{s}.  Berechnet wurde für die Standartabweichung ein Wert von 88.904. Diese Standartabweichung sagt aus, dass die beiden Funktionen überhaupt nicht miteinander verglichen werden können, da die Abweichung sehr gross ist.





>>>>>>> master
=======
Der Vergleich mit der Simulation und der Messung für den sanften Auf- und Abfahren, welches sich im Kapitel \ref{sec:Vergleich_Mess_Sim_sanft_AufAb} im Anhang befindet, zeigt grosse Unterschiede bei den Amplituden auf. Ein grosser Grund dafür ist, dass bei der Simulation das Hoch- und Runterfahren eine Zeitdauer von \SI{0.3}{s} haben, wobei für die Messung diese Zeitdauer \SI{3}{s} beträgt. Zusätzlich ist bei der Simulation während \SI{6}{s} die Spannung auf dem Spitzenwert, beim Laboraufbau sind es eher \SI{7}{s}.  Berechnet wurde für die Standartabweichung ein Wert von 88.904. Diese Standartabweichung sagt aus, dass die beiden Funktionen überhaupt nicht miteinander verglichen werden können, da die Abweichung sehr gross ist.
\todo{kontrollieren wegen satz}
Im visuellen Vergleich mit dem hartem Auf- und Abfahren zeigt das FFT, dass die Frequenzbänder dünner geworden sind und der Peak bei \SI{50}{Hz} grösser. In der Tabelle \ref{tab:Mess_Spannung_AufAb_sanft} sind die höchsten Amplitudenwerte der Frequenzen, die sich in der nähe der Grundschwingung von \SI{50}{Hz} und der fünften Harmonischen \SI{250}{Hz} aufhalten, aufgelistet. \todo{Einfügen Vergleich Simulation}

Der Vergleich mit der Tabelle \ref{tab:Mess_Spannung_AufAb_hart} zeigt, dass die höchsten Peaks nicht mehr bei den sub- und zwischenharmonischen bei den Frequenzen von \SI{49.65}{Hz} und \SI{50.35}{Hz} liegen, sonder bei \SI{49.95}{Hz} und \SI{50.05}{Hz}. Das Verhältnis bei \SI{250}{Hz} zur Grundschwingung ist beim sanftem Auf- und Abfahren kleiner geworden als beim harten Auf- und Abfahren. Wie auch bei \SI{50}{Hz} sind die höchsten Amplituden näher an der harmonische Schwingung. Der Vergleich zur Simulation erkennt man im Anhang.

>>>>>>> master

\subsubsection{Messungen ASM}

Um zu analysieren, wie sich der Thyristorsteller bei einer ohmsch-induktiver Last verhält, wurden die Messungen mit einem Asynchronmotor wiederholt. Auch hier wurden die verschiedenen Ansteuerungsarten, Phasenanschnitt mit 60\textdegree \hspace{0.02cm} und 90\textdegree \hspace{0.02cm}, Schwingungspaket mit Duty Cycle von 0.5 und 0.8 und dem harte und sanfte Auf- und Absteuern durchgeführt. Dabei wurde festgestellt, dass die Schwingungspaketsteuerungen und das harte Auf- und Absteuern sich nicht für einen Asynchronmotor eignen. Der Motor fährt zu schnell Hinauf und Hinunter und befindet sich somit nie in einem geeigneten Stationären Betrieb. 

\subsubsection*{Phasenanschnitt 60\textdegree}

Bei der Abbildung \ref{fig:Mess_ASM_Phas60} verwendete man einen Phasenanschnittswinkel von 60\textdegree.

\begin{figure}[ht!]
	\centering
	\includegraphics[width=\textwidth]{Messung_ASM_Phas_60grad.png}	
	\caption{Messung mit Phasenanschnitt 60\textdegree}\label{fig:Mess_ASM_Phas60}
\end{figure}
<<<<<<< HEAD
<<<<<<< HEAD
 

=======
=======
>>>>>>> master

Anders als beim Phasenanschnitt mit 60\textdegree\hspace{0.02cm} bei ohmschen Lasten, treten bei ohmsch-induktiven Lasten fast keine harmonischen Oberschwingungen. Da diese Peaks nicht im FFT ersichtlich sind, wurden die Amplituden der Oberschwingungen bis zur elften Ordnung auf der Tabelle \ref{tab:Mess_Spannung_ASM_Phas60} aufgeführt. 
<<<<<<< HEAD
\newpage
>>>>>>> master
=======

\newpage
>>>>>>> master
\begin{table}[ht!]
	\centering
	\begin{tabular}{|l|l|l|}
		\hline
		Oberschwingungsordnung & Amplitude {[}V{]} & Verhältnis zur Grundschwingung \\ \hline
		1                      & 181.5519          & 100\%                          \\ \hline
		5                      & 1.1065            & 0.61\%                         \\ \hline
		7                      & 2.8728            & 1.58\%                         \\ \hline
		11                     & 1.4537            & 0.8\%                          \\ \hline
	\end{tabular}
\caption{Amplitudenwerte bei der Frequenzen bei Phasenanschnitt 60\textdegree}\label{tab:Mess_Spannung_ASM_Phas60}
\end{table}

Beim Ansteuern mit dem Winkel von 60\textdegree \hspace{0.02cm} wurde bemerkt, dass die Maschine bereits mit maximaler Drehzahl dreht. Es macht bei den Spannungssignalen keinen Unterschied ob die ASM mit einem Winkel von 60\textdegree \hspace{0.02cm} oder 0\textdegree \hspace{0.02cm} angesteuert wird. Anders als beim Phasenanschnitt mit 60\textdegree\hspace{0.02cm} bei ohmschen Lasten, sind bei ohmsch-induktiven Lasten im FFT keine Oberschwingungen in der nähe der Grundfrequenz ersichtlich. Daher wurden die Amplitudenwerte der Harmonischen Schwingung bis zur 11. Ordnung aufgelistet \ref{tab:Mess_Spannung_ASM_Phas60}. Im Verhältnis zur Grundschwingung hat die 7. Ordnung die maximale Abweichung von 1.58\%.


\newpage
\subsubsection*{Phasenanschnitt 90\textdegree}
Die Abbildung \ref{fig:Mess_ASM_Phas90} zeigt eine Steuerung mit einem Phasenanschnitt von 90\textdegree.

Die Abbildung \ref{fig:Mess_ASM_Phas90}
\begin{figure}[ht!]
	\centering
	\includegraphics[width=\textwidth]{Messung_ASM_Phas_90grad.png}	
	\caption{Messung mit Phasenanschnitt 90\textdegree}\label{fig:Mess_ASM_Phas90}
\end{figure}
<<<<<<< HEAD
<<<<<<< HEAD
 

=======

Anders als beim Phasenanschnitt mit 60\textdegree, beginnt dies ASM bei einem Winkel von 90\textdegree \hspace{0.02cm} zu schwingen. Dieses Schwingen ist gut ersichtlich in der Abbildung \ref{fig:Mess_ASM_Phas90} beim Spannungsverlauf. Dieses verhält sich nicht wie bei einem Phasenanschnitt erwartet konstant. Auch existieren grosse Unterschiede beim FFT mit dem Phasenanschnitt 60\textdegree \hspace{0.02cm} auf der Abbildung \ref{fig:Mess_ASM_Phas60}. Zum einen treten bei 00\textdegree \hspace{0.02cm} Sub- und Zwischenharmonische auf, sowie harmonische Oberwellen. Die Amplituden und das Verhältnis zur Grundschwingung sind für die erste, fünfte und siebte Harmonische, sowie deren Trägerbänder in der Tabelle \ref{tab:Mess_Spannung_ASM_Phas90} aufgelistet. 
\newpage
>>>>>>> master
=======

Anders als beim Phasenanschnitt mit 60\textdegree, beginnt dies ASM bei einem Winkel von 90\textdegree \hspace{0.02cm} zu schwingen. Dieses Schwingen ist gut ersichtlich in der Abbildung \ref{fig:Mess_ASM_Phas90} beim Spannungsverlauf. Dieses verhält sich nicht wie bei einem Phasenanschnitt erwartet konstant. Auch existieren grosse Unterschiede beim FFT mit dem Phasenanschnitt 60\textdegree \hspace{0.02cm} auf der Abbildung \ref{fig:Mess_ASM_Phas60}. Zum einen treten bei 00\textdegree \hspace{0.02cm} Sub- und Zwischenharmonische auf, sowie harmonische Oberwellen. Die Amplituden und das Verhältnis zur Grundschwingung sind für die erste, fünfte und siebte Harmonische, sowie deren Trägerbänder in der Tabelle \ref{tab:Mess_Spannung_ASM_Phas90} aufgelistet. 
\newpage

 


>>>>>>> master
\begin{table}[ht!]
	\centering
	\begin{tabular}{|l|l|l|}
		\hline
		Frequenz {[}Hz{]} & Amplitude {[}V{]} & Verhältnis zur Grundschwingung \\ \hline
		44                & 25.896            & 18.31\%                        \\ \hline
		50                & 141.3976          & 100\%                          \\ \hline
		56                & 20.4508           & 14.46\%                        \\ \hline
		244               & 7.9778            & 5.64\%                         \\ \hline
		250               & 11.6537           & 8.24\%                         \\ \hline
		256               & 1.1655            & 0.82\%                         \\ \hline
		344               & 2.7272            & 1.93\%                         \\ \hline
		350               & 6.8988            & 4.88\%                         \\ \hline
		356               & 2.3509            & 1.66\%                         \\ \hline
	\end{tabular}
\caption{Amplitudenwerte bei der Frequenzen bei Phasenanschnitt 90\textdegree}\label{tab:Mess_Spannung_ASM_Phas90}
\end{table}


Anders als beim Phasenanschnitt mit 60\textdegree, beginnt die ASM bei einem Winkel von 90\textdegree \hspace{0.02cm} zu schwingen. Es ist in der Abbildung \ref{fig:Mess_ASM_Phas90} beim Spannungsverlauf ersichtlich, da die Signale verzerrt erscheinen. Vergleicht man das FFT mit dem Phasenanschnitt von 60\textdegree \hspace{0.02cm} in der Abbildung \ref{fig:Mess_ASM_Phas60}, treten viel mehr Oberschwingungen auf. Dies kommt davon, dass das Signal viel mehr Verzerrungen hat.\\
In der Tabelle \ref{tab:Mess_Spannung_ASM_Phas90} sind die Amplitudenwerte der 1., 5., und 7. Harmonischen mit deren Trägerbänder aufgelistet. Es ist ersichtlich, dass das Verhältnis zur Grundschwingung bei den harmonische Schwingungen viel grösser ist als beim Phasenanschnitt mit einem Winkel von 60\textdegree. 

\newpage
\subsubsection*{Sanftes Auf- und Absteuern}

Die Abbildung \ref{fig:Mess_ASM_Sanft_langsam} zeigt ein sanftes Auf- und Absteuern der Spannung.

\begin{figure}[ht!]
	\centering
	\includegraphics[width=\textwidth]{Messung_ASM_Sanft_langsam.png}	
	\caption{Messung mit sanftem Auf- und Absteuern}\label{fig:Mess_ASM_Sanft_langsam}
\end{figure}


\begin{table}[ht!]
	\centering
	\begin{tabular}{|l|l|l|}
		\hline
		Frequenz {[}Hz{]} & Amplitude {[}V{]} & Verhältnis zur Grundschwingung \\ \hline
		49.85             & 17.3653           & 16.85\%                        \\ \hline
		49.95             & 70.316            & 68.23\%                        \\ \hline
		50                & 103.0639          & 100\%                          \\ \hline
		50.05             & 40.167            & 38.97\%                        \\ \hline
		50.1              & 20.209            & 19.61\%                        \\ \hline
		249.95            & 2.607             & 2.53\%                         \\ \hline
		250               & 1.689             & 1.64\%                         \\ \hline
		250.05            & 2.5084            & 2.43\%                         \\ \hline
	\end{tabular}
\caption{Amplitudenwerte bei der Frequenzen bei sanftem Auf- und Absteuern}\label{tab:Mess_Spannung_ASM_AufAb_sanft}
\end{table}


Bei diesem Steuerungsverfahren sind die harmonischen Schwingungen mit den dazu gehörigen Trägerbänder nur noch minim erkennbar. Die Resultate der Amplitudenwerte und das Verhältnis zur Grundschwingung befinden sich in der Tabelle \ref{tab:Mess_Spannung_ASM_AufAb_sanft}. Die meisten zwischenharmonischen Schwingungen treten um bei der Grundfrequenz von \SI{50}{Hz} auf. Es ist eine Ähnlichkeit zum Widerstand ersichtlich.



\newpage
\subsection{Sparvariante}
Wie im Kapitel \ref{Spar-Ansteuerung} beschrieben, werden bei der Sparvariante nur ein oder zwei Thyristoren angsteuert. Das Spannungs- und Stromsignal soll dabei mehr oder weniger die gleiche Form haben. Da dies bei der Ansteuerung mit einem Thyristor nicht der Fall ist, werden die in der Messauswertung nicht aufgeführt, sondern befinden sich im Anhang im Kapitel \ref{sec:Sparvariante_1Thyristor} Für die Sparvarianten wurden nur neuen Verfahren, sanftes und hartes Auf- und Absteuern, aufgeführt, da hauptsächlich diese von Interesse sind. Die Messungen der sonstigen Ansteuerungsarten befinden sich für den Widerstand im Anhang im Kapitel \ref{sec:Sparvariante_2Thyristoren}.

\subsubsection{Strommessung Sparvariante mit einem Widerstand und zwei Thyristoren}
\subsubsection*{Sanftes Auf- und Absteuern}

\begin{figure}[ht]
	\centering
	\includegraphics[width=\textwidth]{Messung_Widerstand_2Thyristoren_Stroeme.png}	
	\caption{Strommessung mit dem sanften Auf- und Absteuern und zwei Thyristoren}\label{fig:Mess_2Thyristoren_Widerstand_AufAbFahren_langsam_stroeme}	
\end{figure}

\begin{table}[ht]
	\centering
	\begin{tabular}{|l|l|l|l|}
		\hline
		Frequenz {[}Hz{]} & Amplitude Phase 1 {[}A{]}                                                           & Amplitude Phase 2 {[}A{]}                                                           & Amplitude Phase 3 {[}A{]}                                                           \\ \hline
		49.9              & 4.887                                                                               & 3.749                                                                               & 2.354                                                                               \\ \hline
		49.95             & 9.018                                                                               & 7.485                                                                               & 9.618                                                                               \\ \hline
		50                & 9.951                                                                               & 7.881                                                                               & 11.225                                                                              \\ \hline
		50.05             & 5.193                                                                               & 4.31                                                                                & 3.031                                                                               \\ \hline
		50.1              & 3.057                                                                               & 1.522                                                                               & 1.809                                                                               \\ \hline
		149.93            & 1.381                                                                               & 0.525                                                                               & 1.838                                                                               \\ \hline
		150               & 0.899                                                                               & 0.552                                                                               & 1.103                                                                               \\ \hline
		150.2             & 1.386                                                                               & 0.544                                                                               & 1.829                                                                               \\ \hline
		250		          & 0.372                                                                               & 0.273                                                                               & 0.291                                                                               \\ \hline \hline
		Frequenz {[}Hz{]} & \begin{tabular}[c]{@{}l@{}}Verhältnis zur \\ Grundschwingung\\ Phase 1\end{tabular} & \begin{tabular}[c]{@{}l@{}}Verhältnis zur \\ Grundschwingung\\ Phase 2\end{tabular} & \begin{tabular}[c]{@{}l@{}}Verhältnis zur \\ Grundschwingung\\ Phase 3\end{tabular} \\ \hline
		49.9              & 49.11\%                                                                             & 47.57\%                                                                             & 20.97\%                                                                             \\ \hline
		49.95             & 90.62\%                                                                             & 94.98\%                                                                             & 85.68\%                                                                             \\ \hline
		50                & 100\%                                                                               & 100\%                                                                               & 100\%                                                                               \\ \hline
		50.05             & 52.19\%                                                                             & 54.69\%                                                                             & 27\%                                                                                \\ \hline
		50.1              & 30.72\%                                                                             & 19.32\%                                                                             & 16.12\%                                                                             \\ \hline
		149.95            & 13.88\%                                                                             & 6.66\%                                                                              & 16.37\%                                                                             \\ \hline
		150               & 9.03\%                                                                              & 7\%                                                                                 & 9.83\%                                                                              \\ \hline
		150.05            & 13.93\%                                                                             & 6.9\%                                                                               & 16.29\%                                                                             \\ \hline
		250		          & 3.74\%                                                                             & 3.46\%                                                                               & 2.59\%                                                                             \\ \hline
	\end{tabular}
	\caption{Amplitudenwerte bei der Strommessung mit zwei Thyristoren bei sanftem Auf- und Absteuern}\label{tab:Mess_2Thyristoren_Spannung_Widerstand_AufAb_sanft_stroeme}
\end{table}

\newpage
\subsubsection{Spannungsmessung Sparvariante mit einem Widerstand und zwei Thyristoren}
Bei der Sparvariante mit zwei Thyristoren, wurde die dritte Phase überbrückt und direkt auf den Widerstand geführt. Anders als bei den anderen dreiphasigen Ansteuerungen, wird das FFT der drei Phasenspannungen separat aufgezeigt, da diese nicht alle gleich sind. In den Tabellen werden dabei die Amplituden der drei Phasen bei verschiedenen Frequenzen und das Verhältnis zur Grundschwingung aufgelistet. 

Visuell sieht das harte Auf- und Absteuern mit der Sparvariante fast gleich aus wie die mit den drei angesteuerten Thyristoren auf der Abbildung \ref{fig:Mess_Sanft}. Jedoch beim Betrachten des FFTs zeigen sich grosse Unterschiede. Die Werte der Amplituden bei verschiedenen Frequenzen sind auf der Tabelle \ref{tab:Mess_2Thyristoren_Spannung_Widerstand_AufAb_hart} aufgelistet.

Der Vergleich der verschiedenen Phasen zeigt auf, dass die Phasen unterschiedlich belastet sind. Erstmals tritt bei den Messungen die dritte Harmonische auf. Die Auswirkungen dieser Harmonische ist im Kapitel \todo{Kapitel 3te Harmonische} aufgezeigt. Es entstehen bei den Amplituden Unterschiede von fast 30\% zwischen den verschiedenen Phasen auf. 
\newpage
\subsubsection*{Sanftes Auf- und Absteuern}

\begin{figure}[ht!]
	\centering
	\includegraphics[width=\textwidth]{Mess_2Thyristoren_Widerstand_AufAbFahren_langsam.png}	
	\caption{Messung mit dem sanften Auf- und Absteuern und zwei Thyristoren}\label{fig:Mess_2Thyristoren_Widerstand_AufAbFahren_langsam}	
\end{figure}

Wie auch beim hartem Auf- und Absteuern, sehen die Spannungsverläufe auf der Abbildung \ref{fig:Mess_2Thyristoren_Widerstand_AufAbFahren_langsam} visuell sehr ähnlich auch das sanfte Auf- und Absteuern mit drei Thyristoren auf der Abbildung \ref{Mess_1Thyristoren_Widerstand_AufAbFahren_langsam}. Einziger Unterschied ist, das beim Hoch- und Runterfahren die drei Phasen nicht alle gleich hoch sind. Dies resultiert in einem FFT mit unterschiedlichen Amplitudenhöhen für die verschiedenen Phasen. Die Werte des FFTs sind in der Tabelle \ref{tab:Mess_2Thyristoren_Spannung_ASM_AufAb_sanft} aufgelistet. 

\newpage
\begin{table}[ht!]
	\centering
	\begin{tabular}{|l|l|l|l|}
		\hline
		Frequenz {[}Hz{]} & Amplitude Phase 1 {[}V{]}                                                           & Amplitude Phase 2 {[}V{]}                                                           & Amplitude Phase 3 {[}V{]}                                                           \\ \hline
		49.9              & 38.5998                                                                             & 19.6499                                                                             & 34.7131                                                                             \\ \hline
		49.95             & 77.5993                                                                             & 82.1127                                                                             & 60.2946                                                                             \\ \hline
		50                & 191.1857                                                                            & 169.7545                                                                            & 206.7036                                                                            \\ \hline
		50.05             & 125.8716                                                                            & 123.6057                                                                            & 126.0935                                                                            \\ \hline
		50.1              & 42.6127                                                                             & 17.4015                                                                             & 26.3464                                                                             \\ \hline
		149.8             & 12.7189                                                                             & 3.9393                                                                              & 16.14                                                                               \\ \hline
		150               & 2.6765                                                                              & 4.3294                                                                              & 5.5055                                                                              \\ \hline
		150.2             & 9.6611                                                                              & 5.9313                                                                              & 13.152                                                                              \\ \hline \hline
		Frequenz {[}Hz{]} & \begin{tabular}[c]{@{}l@{}}Verhältnis zur \\ Grundschwingung\\ Phase 1\end{tabular} & \begin{tabular}[c]{@{}l@{}}Verhältnis zur \\ Grundschwingung\\ Phase 2\end{tabular} & \begin{tabular}[c]{@{}l@{}}Verhältnis zur \\ Grundschwingung\\ Phase 3\end{tabular} \\ \hline
		49.9              & 20.19\%                                                                             & 11.58\%                                                                             & 16.79\%                                                                             \\ \hline
		49.95             & 40.59\%                                                                             & 48.37\%                                                                             & 29.17\%                                                                             \\ \hline
		50                & 100\%                                                                               & 100\%                                                                               & 100\%                                                                               \\ \hline
		50.05             & 65.84\%                                                                             & 72.81\%                                                                             & 61\%                                                                                \\ \hline
		50.1              & 22.29\%                                                                             & 10.25\%                                                                             & 12.75\%                                                                             \\ \hline
		149.8             & 6.65\%                                                                              & 2.32\%                                                                              & 7.81\%                                                                              \\ \hline
		150               & 1.4\%                                                                               & 2.55\%                                                                              & 2.66\%                                                                              \\ \hline
		150.2             & 5.05\%                                                                              & 3.49\%                                                                              & 6.36\%                                                                              \\ \hline
	\end{tabular}
\caption{Amplitudenwerte bei der Frequenzen mit zwei Thyristoren bei sanftem Auf- und Absteuern}\label{tab:Mess_2Thyristoren_Spannung_Widerstand_AufAb_sanft}
\end{table}


Bei dem Peak der Grundschwingung gibt es eine Abweichung von fast 10\% zwischen den drei Phasen. Dies ist auch bei den anderen Frequenzen der Fall, im Verhältnis zur Grundschwngung der jeweiligen Phase. Einzig bei der dritten Harmonischen ist die Abweichung relativ wenig mit 1.22\%. 

\newpage

\newpage
\subsubsection{Sparvariante mit der ASM und zwei Thyristoren}
Bei der Ansteuerung der ASM eignet sich das harte Auf- und Absteuern nicht, deshalb wird nur das sanfte Auf- und Absteuern aufgezeigt.


Anders als bei den beiden Sparvariante mit dem Widerstand, zeigen sich bei der Sparvariante mit der ASM auf der Abbildung \ref{fig:Mess_2Thyristoren_ASM_AufAbFahren_langsam} im Vergleich zum sanftem Auf- und Absteuern mit drei Thyristoren auf der Abbildung \ref{fig:Mess_ASM_Sanft_langsam} bereits visuelle Unterschiede. So sieht das Hoch- und Runterfahren bei der Sparvariante linearer aus als beim Betrieb mit drei Thyristoren. Bei der Sparvariante mit der ASM und zwei Thyristoren treten ebenfalls Unterschiede zwischen den FFTs der verschiedenen Phasen auf, diese sind in der Tabelle \ref{tab:Mess_2Thyristoren_Spannung_ASM_AufAb_sanft} aufgeführt. 


Im Vergleich zu den Werten des FFTs bei dem sanften Auf- und Absteuern mit drei Thyristoren in der Tabelle \ref{tab:Mess_Spannung_ASM_AufAb_sanft}, ist ersichtlich, dass die Peaks des Trägerbandes um \SI{50}{Hz} im Verhältnis zur Grundschwingung bei der Sparvariante kleiner sind, als bei der Ansteuerung mit drei Thyristoren. Ein weiterer Unterschied ist die dritte Harmonische, welche nur bei der Sparvariante vorkommt. 


\newpage
\subsection{Leistungsfaktor}
Um den Leistungsfaktor für den Phasenanschnitt berechnen zu können, wird die Formeln im Kapitel \ref{sec:Leistungsfaktor} benutzt. Jedoch funktioniert diese Formeln bei der Kombination der verschiedenen Verfahren oder mit dem Hochfahren durch den Phasenanschnitt nicht. Mit folgender Formel kann der Leistungsfaktor ebenfalls berechnet werden:
\begin{equation}
\lambda = \frac{P}{S}
\end{equation}
Um die Wirkleistung zu erhalten, können die Strom- und Spannungswerte multipliziert werden:
\begin{equation}
P = u(t) \cdot i(t)
\end{equation}
Für die Scheinleistung müssen die Effektivwerte der Spannung und des Stromes multipliziert werden:
\begin{equation}
S = U_{rms} \cdot I_{rms}
\end{equation}
Wobei der Effektivwert des Stromes und der Spannung über die gesamte Zeitdauer berechnet wurde. Der Matlabcode dieser Berechnungen befindet sich im Anhang im Kapitel \ref{sec:Leistungsfaktor_Messungen}. 

Die Resultate der Leistungsfaktorberechnungen sind in der Tabelle \ref{tab:Leistungsfaktor_ASM_Widerstand} aufgeführt.

\begin{table}[ht!]
	\centering
	\begin{tabular}{|l|l|}
		\hline
		Ansteuerungsart ASM                                   		& Leistungsfaktor \\ \hline 
		Sanftes Auf- und Absteuern                          		& 0.2947          \\ \hline
		Sanftes Auf- und Absteuern mit 150$\Omega$ Vorwiderstand 	& 0.3493          \\ \hline
		Phasenanschnitt 90\textdegree                               & 0.4879          \\ \hline
		Phasenanschnitt 60\textdegree                               & 0.4155          \\ \hline \hline
		Ansteuerungsart Widerstand                            		& Leistungsfaktor \\ \hline 
		Sanftes Auf- und Absteuern                          		& 0.9987          \\ \hline
		Hartes Auf- und Absteuern                                   & 0.9988          \\ \hline
		Phasenanschnitt 90\textdegree                         		& 0.9953          \\ \hline
		Phasenanschnitt 60\textdegree                         		& 0.999           \\ \hline
	\end{tabular}
\caption{Leistungsfaktor mit verschiedenen Ansteuerungsverfahren bei der ASM und dem Widerstand}\label{tab:Leistungsfaktor_ASM_Widerstand}
\end{table}
Bei der ASM im Leerlauf gilt, je näher der Leistungsfaktor bei 0 ist desto besser. Da der Leistungsfaktor das Verhältnis von Wirk- zu Scheinleistung ist und sich die Maschine im Leerlauf befindet, wird die Wirkleistung nur für die Ummagnetisierungs- und Eisenverluste benötigt. Wenn der Leistungsfaktor höher ist, heisst dies, dass mehr Verlustleistung entsteht. Um dies zu beweisen wurde ein Vorwiderstand mit \SI{150}{\Omega} in den Stromkreis geschaltet. Dabei konnte festgestellt werden, dass sich wie erwartet mit dem Vorwiderstand der Leistungsfaktor erhöht, da zusätzlich Wirkleistung im Widerstand verheizt wird. Wenn der Leistungsfaktor der verschiedenen Ansteuerungsarten analysiert werden, kann klar gesagt werden, dass sich das sanfte Auf- und  Absteuern am besten eignet für die ASM.

Bei Ohmschen Lasten gilt jedoch, je näher der Leistungsfaktor bei 1 ist, desto besser. Bei den Messungen mit den verschiedenen Ansteuerungsarten wurde festgestellt, dass der Leistungsfaktor beim Phasenanschnitt mit einem Winkel von 60\textdegree \hspace{0.02cm} am höchsten ist. Jedoch verbietet die Normen den Gebrauch des Phasenanschnittes mit den Winkeln von 90\textdegree \hspace{0.02cm} wegen den erhöhten Werten der Amplituden. Deshalb macht es Sinn nur das sanfte und harte Auf- und Abfahren zu vergleichen. Bei diesen zwei Ansteuerungsverfahren hat das harte Auf- und Absteuern einen höheren Leistungsfaktor. Jedoch ist der Unterschied von 0.0001 sehr klein und kann praktisch vernachlässigt werden.
 



%\subsubsection{Schwingungspaketsteuerung mit Last in Stern}
%Für die Messung mit der Schwingungspaketsteuerung wurde eine Einschaltzeit von 0.5 Sekunden und eine Ausschaltzeit von 0.2 Sekunden gewählt. Die Ausschaltzeit darf nicht kürzer sein, da die Spannungsverstärkerschaltung und den Thyristorsteller eine Zeitverzögerung darstellen und so die Spannung nicht sofort ein- oder ausgeschaltet wird. Wenn die Ausschaltzeit kürzer ist, geht die Spannung zwischen den Paketen nicht auf \SI{0}{V}. 
%\begin{figure}[ht!]
%	\centering
%	\includegraphics[width=0.7\textwidth]{Schwingungspaket_kurz.png}	
%	\caption{Das Spannungssignal aller Phasen bei Schwingungspaketsteuerung mit FFT}\label{fig:Mess_Schwing_kurz}
%\end{figure}
%
%Das FFT zeigt entgegen den Erwartungen aus der Theorie fast keine Subharmonische auf. Dafür sind Harmonische und Zwischehamrnische sehr ausgeprägt. Sehr gut zu sehen ist die Grundfrequenz von \SI{50}{Hz}, der erste Peak von der linken Seite. Dies ist darauf zurückzuführen, dass nicht direkt ein- und ausgeschaltet wird und so einem Sanft-Anlass ähnelt. Dies dominiert gegenüber dem harten Ein- und Ausschalten, welches die Subharmonische hervorrufen würde.
%\newpage
%\subsection{Phasenanschnittsteuerung mit 2 Thyristoren mit Last in Stern}
%Für die Sparansteuerung wurde ein Winkel von 90\textdegree \hspace{0.02cm} gewählt. 
%
%\begin{figure}[ht!]
%	\centering
%	\includegraphics[width=0.7\textwidth]{2phas_90grad_kurz.png}	
%	\caption{Das Spannungssignal aller Phasen bei Schwingungspaketsteuerung mit FFT}\label{fig:Mess_2phas_kurz}
%\end{figure}
%
%
%\subsection{Phasenanschnittsteuerung mit 1 Thyristor mit Last in Stern}
%Für die Sparansteuerung wurde ein Winkel von 90\textdegree gewählt. 
%
%\begin{figure}[ht!]
%	\centering
%	\includegraphics[width=0.7\textwidth]{1phas_90grad_kurz.png}	
%	\caption{Das Spannungssignal aller Phasen bei Schwingungspaketsteuerung mit FFT}\label{fig:Mess_1phas_kurz}
%\end{figure}




