\begin{abstract}



%Bei empirischen Studien: die Charakteristika der Stichprobe, insbesondere Alter, Geschlecht, Ethnie, Patienten oder gesunde VPn etc., sowie die zentralen Aspekte der verwendeten Methoden und Prozeduren (entspricht dem Methodenteil). Besonders relevant sind solche Informationen, die wahrscheinlich bei der elektronischen Suche berüchsichtigt werden.



%die zentralen Resultate sowie, bei empirischen Studien, die statistischen Signifikanzen und/oder Effektstärken, Konfidenzintervalle u.ä. (entspricht dem Resultateteil)


%die Folgerungen, Implikationen oder Anwendungsmöglichkeiten (entspricht der Diskussion).

%die Fragestellung, wenn möglich in einem einzigen Satz (entspricht der Einführung). Oft wird davor auch knappe Hintergrundinformation zum Forschungsthema gegeben.
\vspace{2ex}
\textbf{Keywords: Oldtimer, Detroit Electrical Car, Batterien}
\end{abstract}	


