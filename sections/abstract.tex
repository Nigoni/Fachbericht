\begin{abstract}

Heutzutage gibt es zwei geläufige Verfahren für die Ansteuerungen von elektrischen Geräten. Eine ist die Phasenanschnittsteuerung, die oft bei ohmschen Lasten verwendet wir. Bei dieser Methode treten mehrfach harmonische Schwingungen auf. Daher sind ihr vom Netzbetreiber einige Grenzen gesetzt. Die zweite Steuerungsart ist die Schwingungspaketsteuerung. Hierbei wird die Leistung während mehreren Netzperioden voll bezogen und danach wieder abgeschaltet. Dieses harte Ein- und Ausschalten des Verbrauchers erzeugt im Netz sub- und zwischenharmonische Schwingungen. Der Nachteil dieser Oberschwingungen sind, Verunreinigungen des öffentlichen Netzes. Sie können das Fehlverhalten von Betriebsmitteln hervor rufen oder sie sogar zerstören.\\ 
In der vorliegenden Arbeit wurde eine dritte Variante entwickelt, um die Nachteile zu minimieren. Sie beinhaltet eine Kombination der beiden Steuerungsverfahren. Dabei wird das sanfte Hoch- und Runterfahren der Leistung mit der Phasenanschnittsteuerung ausgeführt. Die Schwingungspaketsteuerung kommt zu tragen, um die Anzahl Pakete Ein- und Auszuschalten.\\
Es zeigte sich, dass bei der neu entwickelten Methode die harmonischen Oberschwingungen zu einem minimalen Wert absinken. Sie halten somit die festgelegten Vorschriften des Netzbetreibers ein. Je sanfter die Steilheit des Ein- und Ausschalten der Leistung gewählt wurde, desto näher waren die Zwischenharmonischen bei den jeweiligen harmonischen Schwingungen, im Vergleich zur Schwingungspaketsteuerung. Ausserdem waren die Seitenbänder viel dünner als bei der Paketsteuerung. Die Auswirkung dieser sub- und zwischenharmonischen Schwingungen konnten jedoch nicht genau analysiert werden.

 
  
   

%Bei empirischen Studien: die Charakteristika der Stichprobe, insbesondere Alter, Geschlecht, Ethnie, Patienten oder gesunde VPn etc., sowie die zentralen Aspekte der verwendeten Methoden und Prozeduren (entspricht dem Methodenteil). Besonders relevant sind solche Informationen, die wahrscheinlich bei der elektronischen Suche berüchsichtigt werden.



%die zentralen Resultate sowie, bei empirischen Studien, die statistischen Signifikanzen und/oder Effektstärken, Konfidenzintervalle u.ä. (entspricht dem Resultateteil)


%die Folgerungen, Implikationen oder Anwendungsmöglichkeiten (entspricht der Diskussion).

%die Fragestellung, wenn möglich in einem einzigen Satz (entspricht der Einführung). Oft wird davor auch knappe Hintergrundinformation zum Forschungsthema gegeben.
\vspace{2ex}
\textbf{Keywords: Phasenanschnitt, Schwingungspaket, Ohmsche Last}
\end{abstract}	


