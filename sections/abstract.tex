\begin{abstract}

Heutzutage gibt es zwei geläufige Verfahren für eine sinnvolle Ansteuerungen von elektrischen Geräten. Bei ohmschen Lasten wird oft eine Phasenanschnittsteuerung verwendet. Da bei dieser Methode Harmonische Schwingungen auftreten, sind ihr vom Netzbetreiber einige Grenzen gesetzt. Daher weicht man bei solchen Umständen mehrfach auf die Schwingungspaketsteuerung aus. Hierbei wird die Leistung während mehreren Netzperioden voll bezogen und danach wieder abgeschaltet. Dieses harte ein- und ausschalten des Verbrauchers, erzeugt im Netz neben Harmonischen auch Zwischenharmonische Schwingungen.\\
Damit diese Nachteile minimiert werden können, entwickelte man eine dritte Variante. Sie beinhaltet eine Kombination der beiden Steuerungsverfahren. Dabei wird das sanfte Hoch- und Runterfahren der Leistung mit der Phasenanschnittsteuerung ausgeführt, während einer Anzahl Pakete volle Leistung die Schwingungspaketsteuerung zu tragen kommt.\\
Es zeigte sich, dass bei der neu entwickelten Methode die Harmonischen Oberschwingungen zu einem minimalen Wert gesunken sind. Sie halten auch die festgelegten Vorschriften des Netzbetreibers ein. Je sanfter die Steilheit des Ein- und Ausschalten der Leistung gewählt wurde, desto näher waren die Zwischenharmonischen bei den jeweiligen Harmonischen Schwingungen, im Vergleich zur Schwingungspaketsteuerung. Ausserdem waren die Trägerbänder viel dünner als zur Paketsteuerung. 

 
  
   

%Bei empirischen Studien: die Charakteristika der Stichprobe, insbesondere Alter, Geschlecht, Ethnie, Patienten oder gesunde VPn etc., sowie die zentralen Aspekte der verwendeten Methoden und Prozeduren (entspricht dem Methodenteil). Besonders relevant sind solche Informationen, die wahrscheinlich bei der elektronischen Suche berüchsichtigt werden.



%die zentralen Resultate sowie, bei empirischen Studien, die statistischen Signifikanzen und/oder Effektstärken, Konfidenzintervalle u.ä. (entspricht dem Resultateteil)


%die Folgerungen, Implikationen oder Anwendungsmöglichkeiten (entspricht der Diskussion).

%die Fragestellung, wenn möglich in einem einzigen Satz (entspricht der Einführung). Oft wird davor auch knappe Hintergrundinformation zum Forschungsthema gegeben.
\vspace{2ex}
\textbf{Keywords: Phasenanschnitt, Schwingungspaket, Ohmsche Last}
\end{abstract}	


